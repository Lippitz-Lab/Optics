\renewcommand{\lastmod}{December 20, 2023}
\renewcommand{\chapterauthors}{Markus Lippitz}

\chapter{Interference}


\goal{By the end of this chapter you should be able to explain and experimentally demonstrate the operation of an interferometer.}


\section{Overview}

Interference describes phenomena resulting from the superposition of two or more waves. We have already seen such phenomena in previous chapters: diffraction of light at apertures and Fourier optics in general, or rotation of polarization states in optically active media. In this chapter we will discuss interference phenomena in more detail. A prerequisite for interference is coherence, which is the topic of the next chapter. Here we assume that all waves are monochromatic and coherent, and only in the next chapter we discuss what this means.

This chapter consists of two parts: first we discuss the most common types of interferometers and their applications, then we move on to interference in planar stacks of dielectric materials, as used in dielectric filters.


\section{Interference in general}

In chapter \ref{chap:dielectrics} we have defined the intensity  $I$  of a wave as the temporal average of the Poynting vector projected on a reference surface:
\begin{equation}
    I = \braket{\bS \cdot \boldsymbol{n}}_T = \frac{c \, n \, \epsilon_0}{2} \, | E_0 |^2 
\end{equation}
 where  $E_0 $ is the amplitude of the electrical field in one of the two orthogonal polarization eigen-states\sidenote{horizontal and vertical linear or RCP and LCP} $\boldsymbol{\hat{p}}$ , i.e.
\begin{equation}
  \bE(\br, t) = E_0 \, \boldsymbol{\hat{p}} \, e^{i (\bk \br - \omega t)}  \quad .
\end{equation}
Interference occurs when $E_0$ results from the superposition of several waves, i.e. $E_0 = \sum E_i$. These waves are summed \emph{coherently}
\begin{equation}
  I \propto \left| \sum_i E_i \right|^2 
\end{equation} 
in contrast to an incoherent sum
\begin{equation}
  I \propto \sum_i \left|  E_i \right|^2  \quad .
\end{equation} 
Waves in the same polarization eigen-state are summed coherently, waves in different polarization eigen-states are summed incoherently. In the latter case, no interference occurs.
When only two (complex-valued) wave amplitudes contribute, we can write 
\begin{align}
  I =  & \frac{c \, n \, \epsilon_0}{2} \left| E_1 + E_2 \right|^2 \\
  =  & \frac{c \, n \, \epsilon_0}{2}  \left\{
  |E_1|^2 +  |E_2|^2 + 2 |E_1| \, |E_2| \, \cos \Delta \phi
  \right\} \\
  = & I_1 + I_2 + 2 \sqrt{I_1 I_2} \cos \Delta \phi
\end{align} 
where $\Delta \phi$ is the phase difference $\phi_2 - \phi_1$ between the waves with $E_i = |E_i| e^{i \phi_i}$. The detected intensity changes periodically with the phase difference  $\Delta \phi$. When both waves have the same intensity, then 
\begin{equation}
  I = 2 I_0 (1 + \cos \Delta \phi ) = 4 I_0 \cos^2 \Delta \phi /2  \quad .
\end{equation}


\section{Michelson interferometer}

In a Michelson interferometer one and the same beam splitter splits and recombines the waves that travel along the two arms. After recombination, they have acquired a phase difference $\Delta \phi$ due to a path length difference or a difference in index of refraction of the medium in the interferometer arm.

\begin{marginfigure}
  \inputtikz{\currfiledir michelson}
  \caption{Michelson interferometer}
\end{marginfigure}

The Michelson interferometer has two outputs: one through the fourth facet of the beam splitter, and another one in reverse direction of the incoming beam. The beam-splitter is reciprocal and loss-free, i.e. its (complex valued) refection and transmission coefficients $r$ and $t$ do not depend on the direction of travel, and theirs squares add up to one: $|r|^2+ |t|^2 = 1$. The fields at the outputs are
\begin{align}
  E_a = & E_0 \, \left( r \, t  + t \, r \, e^{i \Delta \phi} \right) \\
  E_b = & E_0 \, \left( r \, r  + t \, t  \, e^{i \Delta \phi} \right)   \quad .
\end{align}
Energy conservation requires that 
\begin{equation}
  |E_a|^2 +  |E_b|^2 =  |E_0|^2
\end{equation}
independent of $\Delta \phi$. This is fulfilled when 
\begin{equation}
  \frac{r}{t} = - \frac{r^\star}{t^\star} \quad.
\end{equation}
This means that  the phases of refection and transmission coefficients differ by 90 degrees, i.e.
\begin{equation}
  r = |r| e^{i \alpha} \quad \text{and} \quad  t = |t| e^{i (\alpha \pm \pi/2) } \quad .
\end{equation}
The Fresnel coefficients of a single interface fulfill this requirement, as do all combinations of interfaces and materials. When one output of the interferometer is at its maximum, the other is at its minimum. Total power is conserved.


\section{Mach-Zehnder interferometer}

In a Mach-Zehnder interferometer, one beam splitter separates the waves and a second beam splitter recombines them. Again, we have two output ports of the interferometer. In a Mach-Zehnder interferometer, they are more easily accessible than in a Michelson interferometer because the incoming beam takes a different path. When the path length in both arms is the same ($\Delta \phi = 0$) the symmetric output takes all power, i.e., each beam is transmitted and reflected once.

\begin{marginfigure}
  \inputtikz{\currfiledir mach_zehnder}
  \caption{Mach-Zehnder interferometer}
\end{marginfigure}


Mach-Zehnder interferometers are used in optical telecommunications to modulate the intensity of a beam in an optical fiber. By slightly changing the refractive index in one arm of the interferometer, a phase difference is built up that switches one output from bright to dark. The power of the second output is then dumped.

\section{Sagnac interferometer}
The Sagnac interferometer is an example of a common-path interferometer. Both beams travel the same path, but in reverse direction. When both beams have seen the same index of refraction, the total power is reflected back through the input port, as in this direction, each beam is reflected and transmitted once by the only beam splitter of the interferometer.


\begin{marginfigure}
  \inputtikz{\currfiledir sagnac}
  \caption{Sagnac interferometer}
\end{marginfigure}


Only a few effects are capable of causing an effective path length difference for the counterpropagating beams: The interferometer could rotate, magneto-optical effects of non-reciprocal media could be involved, or short laser pulses could interact with a rapidly changing medium. The rotating Sagnac interferometer is the basis of a laser gyroscope. The phase difference between the beams depends on the angular velocity of rotation $\omega$, the area $A$ of the interferometer, and the dot product between their directions:
\begin{equation}
  \phi = \frac{8 \pi}{\lambda c} \, \boldsymbol{\omega} \cdot  \boldsymbol{A}  \quad .
\end{equation}
A fiber-loop increases the effect.


\section{Fabry-Perot interferometer}


\begin{marginfigure}
  \inputtikz{\currfiledir fabry_perot}
  \caption{Fabry-Perot interferometer. The mirrors reflect less than 100\% of the light.}
\end{marginfigure}


All of the examples above were two-wave interference. The Fabry-Perot interferometer is an example of multiwave interference. It consists of two parallel, partially reflecting mirrors. The transmitted beam is composed of several waves: the directly transmitted wave, a wave with one round trip, a wave with two round trips, and so on. Compare to the direly transmitted wave $E_1$, all other waves acquire for each round trip a complex factor $h$ that contains two reflections at the mirrors and a phase lag due to the distance between the mirrors.
As equation, this reads
\begin{equation}
  E_\text{out} = E_1 \left( 1 + h + h^2 + \dots  \right) = \frac{ E_1 }{1 - h}  \quad . \label{eq:6_FP_Eout}
\end{equation}
The intensity is thus
\begin{equation}
  I = \frac{I_0}{| 1 - h |^2} = \frac{I_0}{(1 - |h|)^2 + 4 |h| \sin^2 \phi/2}
\end{equation}
with $h = |h| e^{i \phi}$. This can be written as
\begin{equation}
  I = \frac{I_\text{max}}{1 + (2 \mathcal{F}/ \pi)^2  \sin^2 \phi/2}
\end{equation}
with the \emph{finesse} $\mathcal{F}$
\begin{equation}
  \mathcal{F} = \frac{\pi \sqrt{|h|}}{1 - |h|}  \quad .
\end{equation}
The higher the quality of the mirrors, the better their reflectivity, the closer $|h|$ gets to one, and the higher the finesse.

The Fabry-Perot interferometer is a spectral filter. Let us assume that the reflectivity $r$ of the mirrors does not depend on the wavelength. The distance $d$ between the mirrors leads to a wavelength-dependent phase shift $\phi$
\begin{equation}
  \phi = 2 \, d \, k = \frac{4 \pi d}{\lambda} =  \frac{4 \pi d  \nu}{c}
\end{equation}
with the optical frequency $\nu$. We get a peak in transmission when $\phi/2  = n \pi$ or when the frequency of the optical wave changes by 
\begin{equation}
  \Delta \nu_\text{FSR} = \frac{c}{2d}
\end{equation}
which is called the \emph{free spectral range}. The FWHM of a peak is as phase $ \delta \phi $ or frequency $ \delta \nu$ 
\begin{equation}
  \delta \phi = \frac{2 \pi }{\mathcal{F}} \qquad\qquad \delta \nu = \frac{c}{2 d \mathcal{F}}  \quad .
\end{equation}
The finesse $\mathcal{F}$ can be written as
\begin{equation}
  \mathcal{F} = \frac{ \Delta \nu_\text{FSR}}{ \delta \nu}
\end{equation}
and defines the usefulness of the interferometer by free span between the peaks per peak width. The Fabry-Perot interferometer is a narrow spectral filter, but its transmission function has repeated maxima. One must therefore know in advance that the expected signal is spectrally narrower than the free spectral range. Or you can cascade several interferometers of different fineness, with the first one acting as a pre-filter for the others. By tuning the distance between the mirrors, the Fabry-Perot interferometer can be used as a spectrometer.


\section{Optics of layered media: Transmission and scattering matrix}

Multiple beam interference is tedious because you have to keep track of the multiple reflection and transmission paths. In the case of the Fabry-Perot interferometer, we could identify the series and have an analytical solution. With three or more layers, this becomes difficult. One way out is the T-matrix method. It is a versatile technique for studying the optical properties of layered media, which are stacks of unstructured films of materials with different dielectric functions. These can be dielectric materials leading to e.g. Bragg reflections and dielectric filters, or metal films leading to surface plasmons. We will discuss transmission and reflection of such stacks.

In a layered medium, a wave traveling through the stack of layers is partially reflected and partially transmitted at each interface. The multiple reflections interfere with each other. To keep track of this, we use in each layer a combined wave traveling in the $+z$ direction and one traveling in the $-z$ direction. These waves mix at interfaces. This formalism is described in chapter 7 of \cite{SalehTeich1991} and \cite{Yeh2005}. A similar formalism with an $E$ and $B$ field traveling in the same direction is described in \cite{Pedrotti2008} and \cite{Macleod2001}.

\begin{marginfigure}
\inputtikz{\currfiledir T-matrix}
\caption{The operation of the transmission matrix
\label{fig:6_T_matrix}}
\end{marginfigure}

Let us assume that we have left of the interface a wave traveling to the right ($+z$ direction) of amplitude $U_1^+$, and one wave traveling to the left of amplitude $U_1^-$. On the right side of the interface, we get the amplitudes $U_2^\pm$ by multiplication with a \emph{transmission} or \emph{transfer} matrix $\mathbf{M}$
\begin{equation}
\begin{pmatrix}
U_2^+ \\ U_2^-
\end{pmatrix}
= 
\begin{pmatrix}
A & B \\ C & D \\
\end{pmatrix}
\cdot
\begin{pmatrix}
U_1^+ \\ U_1^-
\end{pmatrix}
%
= \mathbf{M}
\begin{pmatrix}
U_1^+ \\ U_1^-
\end{pmatrix} \quad . \label{eq:6_def_T_matrix}
\end{equation}
Below we will derive transmission matrices $\mathbf{M}_i$ for every interface and the homogeneous space in between. The full stack can then be described by a product matrix, multiplying together all partial matrices $\mathbf{M}_i$ along the stack
\begin{equation}
\mathbf{M}_\text{total} = \mathbf{M}_n \cdot  \mathbf{M}_{n-1} \cdots\mathbf{M}_2 \cdot  \mathbf{M}_{1} \quad . 
\end{equation}
This is a very convenient feature of the transmission matrix.
Note that we label the interactions from left to right with $1$ to $n$, but the matrices are multiplied from right to left, as mathematics has it origin in Arabic culture.

An inconvenient feature of the transmission matrix is that its matrix element have no direct physical meaning. The problem is that we multiply on the matrix a vector that is half an input, half an output of this interface. We know what comes out (travels to the left), and the matrix should tell us what comes in from the other side. In this sense, the related \emph{scattering} matrix $\mathbf{S}$ is closer to physical meaning:
\begin{equation}
\begin{pmatrix}
U_2^+ \\ U_1^-
\end{pmatrix}
= 
\begin{pmatrix}
t_{12} & r_{21}  \\ r_{12} & t_{21}
\end{pmatrix}
\cdot
\begin{pmatrix}
U_1^+ \\ U_2^-
\end{pmatrix}
%
= \mathbf{S}
\begin{pmatrix}
U_1^+ \\ U_2^-
\end{pmatrix} \quad . 
\end{equation}
The scattering matrix connects waves traveling towards the interface with those traveling away from the interface. The entries $t_{ij}$ and $r_{ij}$ are the transmission and reflection coefficients for the amplitudes of the waves traveling from $i$ to $j$ (i.e. $12$ is traveling towards the right, $+z$ direction). However, for the scattering matrix $\mathbf{S}$, the full stack can not be calculated by multiplying together all partial matrices.

\begin{marginfigure}
%\includegraphics[width=50mm]{\currfiledir S-matrix.png}
\inputtikz{\currfiledir S-matrix}

\caption{The operation of the scattering matrix
\label{fig:6_S_matrix}}
\end{marginfigure}

It is therefore convenient to switch between both representations, derive the scattering matrix $\mathbf{S}$ for each situation, and then convert into a transmission matrix $\mathbf{M}$. The relations are\sidenote{\cite{SalehTeich1991}  eq. 7.7}
\begin{align}
\mathbf{M} =  &
\begin{pmatrix}
A & B \\ C & D \\
\end{pmatrix}
=
\frac{1}{t_{21}}
\begin{pmatrix}
t_{12} t_{21} - r_{12}r_{21} & r_{21} \\ - r_{12} & 1 \\
\end{pmatrix} \label{eq:6_M_from_S}
\\
\mathbf{S} =  &
\begin{pmatrix}
t_{12} & r_{21}  \\ r_{12} & t_{21}
\end{pmatrix}
=
\frac{1}{D}
\begin{pmatrix}
AD - BC & B \\ -C & 1 \\
\end{pmatrix} \label{eq:6_S_from_M}
\end{align}
as long as $D$ or $t_{21}$ are not zero.


The transmission in backward direction $t_{21}$ is thus the reciprocal of the $D$-element of $\mathbf{M}_\text{total} $. The transmission in forward direction is 
\begin{equation}
t_{12} = \frac{\text{det } \mathbf{M}_\text{total} }{D}
\end{equation}
and similar for the reflection from the front side
\begin{equation}
r_{12} = - \frac{C }{D} \quad . 
\end{equation}




\section{Electrical fields}

We need to define the physical meaning of the amplitudes $U_i^\pm$ to be able to calculate the reflection ($r_{ij}$) and transmission ($t_{ij}$) coefficients. We assume plane waves 
\begin{equation}
\mathbf{E} \, e^{i (\mathbf{k}  \cdot \mathbf{r} - \omega t)}
=
\mathbf{\hat{E}} \, U \, e^{i \, k_z z} \, e^{i \, k_x x} \, e^{-i \omega t}
\end{equation}
where the wave vector $\mathbf{k} $ lies in the $xz$-plane, $U$ defines the amplitude of the wave and $\mathbf{\hat{E}} $  the polarization direction.
With   the full length of the wave vector in vacuum $k_0 = 2 \pi / \lambda$ and the refractive index $n$ of the medium we get
\begin{equation}
k_{z}^2 + k_{x}^2  = n^2 \, k_0^2  \quad . 
\end{equation}
The polarization directions are
\begin{equation}
\mathbf{\hat{E}}^{(s)} = \begin{pmatrix}
 0 \\ 1 \\ 0 \\
\end{pmatrix}
\quad 
\text{and}
\quad
\mathbf{\hat{E}}^{(p)} =\frac{1}{n \, k_0} \begin{pmatrix}
\pm k_z \\ 0 \\  k_x  \\
\end{pmatrix} \quad . \label{eq:6_Esp_def}
\end{equation}
The $\pm$-sign takes the sign of the direction of travel, see Fig. 2.2 in \cite{Novotny-Hecht2012}. Note that with this definition we have $|\mathbf{\hat{E}}| = 1$, which differs from problem 12.4 in \cite{Novotny-Hecht2012}.

The left and right traveling waves are thus
\begin{equation}
\mathbf{E}^+ = \mathbf{\hat{E}} \, U^+ \, \, e^{+ i \, k_z z}
\quad
\text{and}
\quad
\mathbf{E}^- = \mathbf{\hat{E}} \, U^- \, \, e^{- i \, k_z z}
\end{equation}
where we have split off the global term $ e^{i \, ( k_x x - \omega t)}$.

\section{Propagation matrix}

Before we come to interfaces, let us discuss the transmission matrix of a homogeneous material layer $j$ of thickness $d_j$ and (complex) refractive index $n_j$. Relevant is the $z$-component of the (complex) wave vector $k_{z,j}$. Note that we do \emph{not} use the sign of  $k_{z,j}$ to describe the direction of travel.
Independent of the propagation direction, each wave sees a reflection coefficient $r=0$ and a (complex) transmission coefficient $t$
\begin{equation}
t = t_{12} = t_{21} = e^{+ i \, k_{z,j} \, d_j } \quad . 
\end{equation}
The transmission matrix of a homogeneous medium is thus
\begin{equation}
\mathbf{M} = 
\begin{pmatrix}
e^{+i \, k_{z,j} \, d_j } & 0 \\0 & e^{-i \, k_{z,j} \, d_j } \\
\end{pmatrix} \quad . 
\label{eq:6_M_prob}
\end{equation}


\section{Interface matrix}

The transmission and reflection coefficients of an interface are the Fresnel coefficients $r$ and $t$ for s and p polarization, as defined in chapter \ref{chap:dielectrics}, especially  $r^s$ and $r^p$ differ at normal incidence by a factor of $-1$. We assume non-magnetic materials ($\mu = 1$). Independent of polarization dirction, the forward  and backward coffeicients are related:
\begin{align}
 r_{12} = &   - r_{21}\\
 t_{12} = &   \frac{k_{z,1}}{k_{z,2}}  \,  t_{21}
\end{align}
so that
\begin{equation}
  t_{12} t_{21} - r_{12} r_{21} = t_{21}^2 \frac{k_{z,1}}{k_{z,2}} + r_{21}^2 = 1 \quad . 
  \end{equation}


% We follow here \cite{Novotny-Hecht2012}, who follow \cite{BornWolf2002}, especially in the direction of the field vectors, see Fig. 2.2 in \cite{Novotny-Hecht2012}. In this definition,  $r^s$ and $r^p$ differ at normal incidence by a factor of $-1$. We assume non-magnetic materials ($\mu = 1$) and get for a wave traveling from medium 1 towards medium 2
% \begin{align}
%  r_{12}^s = & \frac{k_{z,1} - k_{z,2}}{k_{z,1} + k_{z,2}}  = - r_{21}^s\\
%  t_{12}^s = & \frac{2 \, k_{z,1}}{k_{z,1} + k_{z,2}} =  \frac{k_{z,1}}{k_{z,2}}  \,  t_{21}^s\\
%   r_{12}^p = & \frac{\epsilon_2	 k_{z,1} - \epsilon_1 k_{z,2}}
% 				  {\epsilon_2 k_{z,1} + \epsilon_1 k_{z,2}}  = - r_{21}^p\\
%   t_{12}^p = & \frac{2 \sqrt{\epsilon_1 \epsilon_2}	 \,k_{z,1} }
% 				  {\epsilon_2 k_{z,1} + \epsilon_1 k_{z,2}}  = \frac{k_{z,1}}{k_{z,2}}  \,  t_{21}^p \quad . 
% \end{align}
% We could also write these coefficients in terms of angle of incidence $\theta$ with
% \begin{equation}
%  \theta = \arcsin \frac{k_x}{n k_0} = \arcsin \sqrt{1 - \left( \frac{k_z}{n k_0} \right)^2 } \quad . 
% \end{equation}
% This would also hold in the case of evanescent waves ($k_x > n k_0$) when we allow complex angles $\theta$. We nowhere need that $\theta$ is a geometrical angle. We only need that $n \sin \theta$ is the same for all layers.

With eq.~\ref{eq:6_M_from_S} we get for both polarization directions the transmission matrix
\begin{equation}
\mathbf{M}_{12} = \frac{1}{t_{21}} 
\begin{pmatrix}
1 & r_{21} \\ r_{21} & 1 \\
\end{pmatrix} \quad .
\end{equation}
Note that the transmission matrix from medium 1 to medium 2 uses the Fresnel coefficients of the backwards direction!
We can write this without referring to Fresnel coefficients as\sidenote{In problem 12.4 in \cite{Novotny-Hecht2012} the leading $1/\eta$ seems to be missing!} (see also appendix at the end of this chapter)
\begin{equation}
\mathbf{M}_{12} 
=\frac{ 1}{2 \eta }
\begin{pmatrix}
1 + \kappa & 1  -\kappa \\  1  - \kappa  & 1 + \kappa \\
\end{pmatrix} \label{eq:6_M_kappa}
\end{equation}
with 
\begin{equation}
\kappa = \eta^2 \,
\frac{  k_{z,1} }{ k_{z,2}}
\quad
\text{and}
\quad
\eta^s = 1 \quad \text{or} \quad \eta^p = \sqrt{ \frac{\epsilon_2}{\epsilon_1} } \quad . 
\end{equation}
The factors $\eta$ in front of the transmission matrix $\mathbf{M}_{12} $ can be collected in front of the total transmission matrix $\mathbf{M}_\text{total}$, in case one is not interested in the distribution of the fields inside the stack. Then, all $\eta^p$ collapse into $\sqrt{\epsilon_\text{first} / \epsilon_\text{last}}$, which is equal to one in case the terminating half-spaces of the layered medium have both the same dielectric constant. 


\section{Example: Fabry-Perot etalon}

A simple way to construct a Fabry-Perot interferometer is a glass plate with parallel surfaces. Each surface acts as (weak) mirror. Let us use the T-matrix method to calculate its transmission properties. We have three media, where medium 2 is glass and media 1 and 3 are air. The total transmission matrix is thus
\begin{equation}
  \mathbf{M}_\text{total}  = \mathbf{M}_{23} \, \mathbf{M}_\text{prop,2} \, \mathbf{M}_{12}   \quad . 
\end{equation}
Inserting the definition from above and using $r_{12} = r_{32} = - r_{21}$ and $t_{12} = t_{32} = (k_{z,3} / k_{z,2}) \, t_{21}$ we get
\begin{equation}
  \mathbf{M}_\text{total}
  = 
  \frac{k_{z,1}}{k_{z,2}} 
  \begin{pmatrix}
  1 & -r_{21} \\ -r_{21} & 1 \\
  \end{pmatrix} 
\, 
\begin{pmatrix}
  e^{+i \, \phi_2 } & 0 \\0 & e^{-i \, \phi_2 } \\
  \end{pmatrix}
\,
\begin{pmatrix}
1 & r_{21} \\ r_{21} & 1 \\
\end{pmatrix} 
\end{equation}
with $\phi_2 = k_{z,2} \, d_2$. The overall transmission is 
\begin{equation}
  t = \frac{\text{det } \mathbf{M}_\text{total} }{D} = 
  \frac{(1 - r_{21}^2)^2   e^{+ i \, \phi_2}} {  1 - r_{21}^2 \, e^{+i \,2 \phi_2 }}
  \end{equation}
which agrees with eq. \ref{eq:6_FP_Eout} using $h = r_{21}^2 \, e^{+i \,2 \phi_2 }$.

\begin{marginfigure}
  \inputtikz{\currfiledir FP-transmission}
  \caption{Transmission of a 1 \textmu m  thick glass layer in air. }
\end{marginfigure}

\section{Quarter-wave film as anti-reflection coating}

The most simple anti-refection coating for an interface between two media of refractive index $n_1$ and $n_3$ is a film of refractive index $n_2 = \sqrt{n_1 \, n_3}$ of thickness $d = \lambda / 4 = \lambda_0 / (4 n_2)$. In this case
\begin{equation}
  \mathbf{M}_\text{total}
  = 
  \frac{k_{z,1}}{k_{z,2}} 
  \begin{pmatrix}
  1 & r_{32} \\ r_{32} & 1 \\
  \end{pmatrix} 
\, 
\begin{pmatrix}
  i & 0 \\0 & -i \\
  \end{pmatrix}
\,
\begin{pmatrix}
1 & r_{21} \\ r_{21} & 1 \\
\end{pmatrix}  \quad . 
\end{equation}
The total reflection vanishes when the $B$ and $C$ elements of the transmission matrix vanish, which is the case, when
\begin{equation}
  r_{32} = r_{21} \quad .
\end{equation}
This holds for perpendicular incidence at the design wavelength. At other wavelength the reflection increases, up to the value of an uncoated $n_1$--$n_3$ interface.

\begin{marginfigure}
  \inputtikz{\currfiledir V-coating}
  \caption{Quarter-wave AR coating on air-glass interface.}
\end{marginfigure}


A spectrally broader antireflective coating requires multiple layers with at least two different refractive indices. Designing such multilayer stacks is an art in itself. Some details are discussed in \cite{Melles_griot_filter}.



\section{Distributed Bragg reflector (DBR) }

An example of a multilayer coating is the distributed Bragg reflector (DBR).  It consists of alternative layers of refractive indices $n_1$ and $n_2$, with thicknesses $d_1$ and $d_2$, embedded in a medium of index $n_1$. The difference between $n_1$ and $n_2$ will be small because we have to stack similar materials. Each surface will reflect only a little. But we let the many single reflections add up coherently, in phase, but with a good choice of the distances $d_i$. In this way a very large total reflectivity can be achieved. These DBRs are used as cavity mirrors in semiconductor lasers.

\begin{marginfigure}
  \inputtikz{\currfiledir DBR}
  \caption{Transmission of 10 pairs of air and glass.}
\end{marginfigure}

The transmission matrix of one pair of layers is
\begin{equation}
  \mathbf{M}_\text{pair}
  = 
  \mathbf{M}_\text{prop,1}    \mathbf{M}_{2,1}  \mathbf{M}_\text{prop,2}    \mathbf{M}_{1,2} \quad . 
\end{equation}
The full stack consists  of $N$ pairs
\begin{equation}
  \mathbf{M}_\text{full}
  = 
  \mathbf{M}_\text{pair}^N     \quad . 
\end{equation}
\cite{SalehTeich1991} show steps of the analytical solution in chapter 7.1. One finds \emph{stop bands}, i.e., spectral regions of high reflectivity, around the Bragg frequency $\nu_B$ at which the round-trip phase in a pair is  $2 \pi$, i.e.
\begin{equation}
  2 \, \frac{2 \pi}{c_0} \, \nu_B \left( n_1 d_1 + n_2 d_2 \right) = 2 \pi \quad . 
\end{equation}


\section{Appendix: derivation of eq. \ref{eq:6_M_kappa}}


We start from 
\begin{equation}
\mathbf{M}_{12} = \frac{1}{t_{21}} 
\begin{pmatrix}
1 & r_{21} \\ r_{21} & 1 \\
\end{pmatrix}
\end{equation}
and abbreviate the Fresnel coefficients as
\begin{align}
  r_{21}^s = & \frac{k_{z,2} - k_{z,1}}{k_{z,1} + k_{z,2}}  = \frac{b - a}{a + b} \\
 t_{21}^s = & \frac{2 \, k_{z,2}}{k_{z,1} + k_{z,2}} =   \frac{2 b \eta }{a + b}   \\
  r_{21}^p = & \frac{\epsilon_1	 k_{z,2} - \epsilon_2 k_{z,1}}
				  {\epsilon_2 k_{z,1} + \epsilon_1 k_{z,2}}  =   \frac{b - a}{a + b}\\
  t_{21}^p = & \frac{2 \sqrt{\epsilon_1 \epsilon_2}	 \,k_{z,2} }
				  {\epsilon_2 k_{z,1} + \epsilon_1 k_{z,2}}  =   \frac{2 b  \eta }{a + b} 
\end{align}
with $a = \epsilon_2 k_{z,1}$, $b =     \epsilon_1 k_{z,2}$ and $\eta = \sqrt{\epsilon_2 / \epsilon_1}$. In the case of s-polarization, the $\epsilon_i$ are ignored / set to one. With this we get
\begin{align}
\mathbf{M}_{12} = & \frac{a+b}{2 b \eta} 
\begin{pmatrix}
1 & (b-a)/(a+b) \\  (b-a)/(a+b) & 1 \\
\end{pmatrix}
= 
 \frac{1}{2 b \eta} 
\begin{pmatrix}
b+a & b-a \\  b-a & b+a \\
\end{pmatrix} \\
= &
 \frac{1}{2  \eta} 
\begin{pmatrix}
1+\frac{a}{b} & 1- \frac{a}{b} \\  1- \frac{a}{b} & 1+\frac{a}{b} \\
\end{pmatrix}
= 
 \frac{1}{2  \eta} 
\begin{pmatrix}
1+\kappa & 1- \kappa \\  1- \kappa & 1+\kappa \\
\end{pmatrix} 
\end{align}
with 
\begin{equation}
\kappa = \frac{a}{b} = \eta^2 \,
\frac{  k_{z,1} }{ k_{z,2}}
\quad
\text{and}
\quad
\eta^s = 1 \quad \text{or} \quad \eta^p = \sqrt{ \frac{\epsilon_2}{\epsilon_1} } \quad .
\end{equation}


%--------------------
\printbibliography[segment=\therefsegment,heading=subbibliography]
