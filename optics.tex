\documentclass[notoc,nofonts,a4paper,twoside,nobib]{tufte-book}
%\documentclass[nofonts,a4paper,twoside]{book}

\usepackage[english]{babel}
\usepackage{currfile,hyperxmp}

\usepackage{filemod}
   \usepackage{dsfont}
\usepackage[
    type={CC},
    modifier={by-sa},
    version={4.0},
    imagewidth = 17mm,
 ]{doclicense} 
  
 \usepackage[T1]{fontenc}
\usepackage[utf8]{inputenc}

 

\usepackage[refsegment=chapter,style=authoryear-comp,natbib=true,url=true,
isbn=false]{biblatex}

\addbibresource{literature.bib}
%\addbibresource{literature_EPC1.bib}
  
 
%rm -rf `biber --cache`


%\AtBeginBibliography{\urlstyle{rm}}

\RequirePackage{fontawesome}

\DeclareFieldFormat{doi}{%
  \ifhyperref
    {\href{http://doi.org/#1}{\small \faExternalLink}}
    {\nolinkurl{#1}}}

\DeclareFieldFormat{url}{%
  \ifhyperref
    {\href{#1}{\small \faExternalLink}}
    {\nolinkurl{#1}}}
    
\renewbibmacro*{doi+eprint+url}{%   
  \iftoggle{bbx:url}     
    {\iffieldundef{doi}{\usebibmacro{url+urldate}}{}}     
    {}%   
  \newunit\newblock   
  \iftoggle{bbx:eprint}     
    {\usebibmacro{eprint}}     
    {}%   
  \newunit\newblock   
  \iftoggle{bbx:doi}     
    {\printfield{doi}}     
    {}}  

\usepackage{amssymb,amsmath}
\usepackage{mathtools,bm}
 
\usepackage{modiagram}
\usepackage{chemformula}
\usepackage{chemfig}
\renewcommand*\printatom[1]{\ensuremath{\mathsf{#1}}}


\input{tikz_header}
\usetikzlibrary{external}
\tikzexternalize[prefix=tikz_external/]




\usepackage{graphicx}
\setkeys{Gin}{width=\linewidth,totalheight=\textheight,keepaspectratio}


\usepackage{booktabs}
\usepackage{url}
\usepackage{hyperref}

%\usepackage{units}

\usepackage{chemformula}

\usepackage{braket}
\setcounter{secnumdepth}{0}

% citations
%\usepackage{natbib}
%\bibliographystyle{plainnat}
%\setcitestyle{round} 

% pandoc syntax highlighting
%\usepackage{color}
%\usepackage{fancyvrb}



% longtable
\usepackage{longtable,booktabs}
\usepackage{multicol}
\usepackage[normalem]{ulem}

% morefloats
\usepackage{morefloats}

\usepackage{calc}
\usepackage{tcolorbox}


\input{tint_book_header}



\renewcommand{\chaptermark}[1]{\markboth{#1}{}}%


\ifthenelse{\boolean{@tufte@twoside}}
  {\fancyhead[LE]{\thepage\quad{\newlinetospace{Optics}}}%
    \fancyhead[RO]{{\newlinetospace{\leftmark}}\quad\thepage}}%
  {\fancyhead[RE,RO]{{\newlinetospace{c}}\quad\thepage}}
  
 
%\makeatletter
\fancypagestyle{mystyle}{%
\fancyhf{}%
\fancyfoot[L]{%
\begin{minipage}{17mm}
\doclicenseImage
\end{minipage}
\begin{minipage}{90mm}
 \footnotesize
 \doclicenseLongText
\end{minipage}%
}% 
%\fancyfoot[L]{\doclicenseThis}% 
}
%\makeatother

\usepackage{etoolbox}
\patchcmd{\chapter}{\thispagestyle{plain}}{\thispagestyle{mystyle}}{}{}



\hypersetup{
 linktocpage,
  colorlinks,
  citecolor=Maroon,
  filecolor=Maroon,
  linkcolor=RoyalBlue,
  urlcolor=RoyalBlue
}


\usepackage[theme=default-plain,charsperline=62]{jlcode}
\usepackage{siunitx}

%default, default-plain, grayscale, grayscale-plain and darkbeamer.




 
\newcommand{\kapitelname}{Chapter\ }
\newcommand{\chapterauthors}{Markus Lippitz}
\newcommand{\lastmod}{\Filemodtoday{\currfilepath}}


\newcommand{\addtochapter}{%
\vspace*{-12mm}{
\setlength{\parindent}{0pt}
\chapterauthors  \newline \lastmod
}
\vspace*{12mm}
}

\makeatletter
\let\stdchapter\chapter
\renewcommand*\chapter{%
  \@ifstar{\starchapter}{\@dblarg\nostarchapter}}
\newcommand*\starchapter[1]{\stdchapter*{#1}}
\def\nostarchapter[#1]#2{\stdchapter[{#1}]{#2} \addtochapter}
\makeatother

\makeatletter
  \def\my@tag@font{\scriptsize}
  \def\maketag@@@#1{\hbox{\m@th\normalfont\color{gray}\my@tag@font#1}}
  \let\amsmath@eqref\eqref
  \renewcommand\eqref[1]{{\let\my@tag@font\relax\amsmath@eqref{#1}}}
\makeatother

\newcounter{questions}[chapter]

\newenvironment{questions}{
\subsection{\normalsize Test yourself}
\begin{enumerate} \small
\setcounter{enumi}{\value{questions}}
}{
\setcounter{questions}{\value{enumi}}
\end{enumerate} 
}

\newtcolorbox{zusammen}{%
  breakable,
  enhanced jigsaw,
 % borderline west={1pt}{0pt}{black},
  sharp corners,
  %boxrule=0pt,
  %frame hidden,
  left=1ex,right=1ex,
  fonttitle={\bfseries},
  coltitle={black},
  title={Zusammenfassung:\ },
  attach title to upper}
  
  
  \newcommand{\pluto}[1]{%
  %
  \edef\cfd{\currfiledir}%
  \StrGobbleRight{\cfd}{1}[\mystring]%
  %
  \sidenote{%
  \begin{tikzpicture}
  [baseline={([yshift=-2pt]current bounding box.center)}]
  \definecolor{redline}{RGB}{201,61,57}
  \definecolor{redfill}{RGB}{214,102,97}
  \definecolor{blueline}{RGB}{148,91,176}
  \definecolor{bluefill}{RGB}{170,125,192}
  \definecolor{greenline}{RGB}{59,151,46}
  \definecolor{greenfill}{RGB}{107,171,91}
  \path[draw=redline,fill=redfill,line width=0.8pt] (0,-5.4pt) circle (4.4pt);
  \path[draw=blueline,fill=bluefill,line width=0.8pt] (0,0) circle (4.4pt);
  \path[draw=greenline,fill=greenfill,line width=0.8pt] (0,5.4pt) circle (4.4pt);
  \end{tikzpicture} \ \ 
  \href{https://raw.githubusercontent.com/MarkusLippitz/Festkoerper-II/main/\mystring/pluto/#1.jl}{download}
  \ \ 
  \href{https://binder.plutojl.org/v0.19.12/open?url=https\%253A\%252F\%252Fraw.githubusercontent.com\%252FMarkusLippitz\%252FFestkoerper-II\%252Fmain\%252F\mystring\%252Fpluto\%252F#1.jl}{run on binder}
  }}  
  

  \newcommand{\bk}{\boldsymbol{k}}
  \newcommand{\br}{\boldsymbol{r}}
  \newcommand{\bj}{\boldsymbol{j}}
  \newcommand{\bx}{\boldsymbol{x}}
  \newcommand{\by}{\boldsymbol{y}}
  \newcommand{\bz}{\boldsymbol{z}}
  % \newcommand{\bD}{\boldsymbol{\mathcal{D}}}
  % \newcommand{\bE}{\boldsymbol{\mathcal{E}}}
  % \newcommand{\bB}{\boldsymbol{\mathcal{B}}}
  % \newcommand{\bH}{\boldsymbol{\mathcal{H}}}
  % \newcommand{\bP}{\boldsymbol{\mathcal{P}}}
  % \newcommand{\bM}{\boldsymbol{\mathcal{M}}}
  % \newcommand{\bS}{\boldsymbol{\mathcal{S}}}  
  
  \newcommand{\bD}{\boldsymbol{D}}
  \newcommand{\bE}{\boldsymbol{E}}
  \newcommand{\bB}{\boldsymbol{B}}
  \newcommand{\bH}{\boldsymbol{H}}
  \newcommand{\bP}{\boldsymbol{P}}
  \newcommand{\bM}{\boldsymbol{M}}
  \newcommand{\bS}{\boldsymbol{S}}
  \newcommand{\bJ}{\boldsymbol{J}}
  \newcommand{\beps}{\boldsymbol{\epsilon}}


  \newcommand{\lit}[1]{#1}
  \newcommand{\ziel}[1]{#1}
  \newcommand{\wort}[1]{\emph{#1}}

  \newcommand{\goal}[1]{{\setlength{\parindent}{0pt}\emph{#1}}}

\usepackage[titletoc]{appendix}
%\renewcommand{\appendixname}{Anhang}
%\renewcommand{\appendixtocname}{Anhang}
%\renewcommand{\appendixpagename}{Anhang}

%\includeonly{preface}
%\includeonly{1_rays/1_rays}
%\includeonly{2_Gauss/2_gauss}
%\includeonly{4_dielectrics/4_dielectrics}
%\includeonly{5_polarization/5_polarization}
%\includeonly{6_interference/6_interference}
%\includeonly{7_coherence/7_coherence}


 
\begin{document}
 
  \tikzexternaldisable


\title{Optics}

\author{Markus Lippitz}
\date{\today}


\maketitle


\newpage
\thispagestyle{empty}

% \hfill

% \vfill

% \noindent \textit{cite as}\\
% \noindent Lippitz, Markus, 2023.  \\
% \noindent Festkörperphysik II - Skript zur Vorlesung (Sommer 2023). Zenodo. \\
% \noindent \url{https://doi.org/10.5281/zenodo.8279873}
% %

\tableofcontents

%\renewcommand{\lastmod}{\ \ }
\renewcommand{\chapterauthors}{\ \ }

\chapter*{Preface}

These are the lecture notes for my lecture on optics. The lecture is aimed at students in the third year of the bachelor programme. It follows the idea of \cite{SalehTeich1991}: we start with very simple models to describe light and gradually increase the complexity but also the power of the model. We start with ray optics and include geometrical optics and lens aberrations. We then move on to scalar waves, introducing Gaussian beams and Fourier optics. The next step is vectorial electromagnetic waves, which allow us to take into account material properties and birefringence. Finally, we come to quantum optics and describe light as a stream of photons.



These notes are 'work in progress', and probably never really finished. If you find mistakes, please tell me. I am also always interested in other sources covering these topics.
The most current version of the lecture notes can be found at github\sidenote{\url{https://github.com/MarkusLippitz}}. There you also find the material for the tasks. I have put everything under a CC-BY-SA license (see footer). In my words: feel free to do with it whatever you like. If you make your work available to the public, mention me and use a similar license. 


The lecture notes are typeset using the LaTeX class 'tufte-book' by Bil Kleb, Bill Wood, and Kevin Godby\sidenote{\href{https://tufte-latex.github.io/tufte-latex/}{tufte-latex}}, which  approximates the work of Edward Tufte\sidenote{\href{https://www.edwardtufte.com/}{edwardtufte.com}}. I applied many of the modifications introduced by Dirk Eddelbuettel in the 'tint' R package\sidenote{\href{https://dirk.eddelbuettel.com/code/tint.html}{tint: Tint is not Tufte}}. For the time being, the source is LaTeX, not markdown.

\vspace{2\baselineskip}

Markus Lippitz \\ Bayreuth, September 18, 2023



\renewcommand{\lastmod}{\ \ }
\renewcommand{\chapterauthors}{\ \ }

\chapter*{Preface}

These are the lecture notes for my lecture on optics. The lecture is aimed at students in the third year of the bachelor programme. It follows the idea of \cite{SalehTeich1991}: we start with very simple models to describe light and gradually increase the complexity but also the power of the model. We start with ray optics and include geometrical optics and lens aberrations. We then move on to scalar waves, introducing Gaussian beams and Fourier optics. The next step is vectorial electromagnetic waves, which allow us to take into account material properties and birefringence. Finally, we come to quantum optics and describe light as a stream of photons.



These notes are 'work in progress', and probably never really finished. If you find mistakes, please tell me. I am also always interested in other sources covering these topics.
The most current version of the lecture notes can be found at github\sidenote{\url{https://github.com/MarkusLippitz}}. There you also find the material for the tasks. I have put everything under a CC-BY-SA license (see footer). In my words: feel free to do with it whatever you like. If you make your work available to the public, mention me and use a similar license. 


The lecture notes are typeset using the LaTeX class 'tufte-book' by Bil Kleb, Bill Wood, and Kevin Godby\sidenote{\href{https://tufte-latex.github.io/tufte-latex/}{tufte-latex}}, which  approximates the work of Edward Tufte\sidenote{\href{https://www.edwardtufte.com/}{edwardtufte.com}}. I applied many of the modifications introduced by Dirk Eddelbuettel in the 'tint' R package\sidenote{\href{https://dirk.eddelbuettel.com/code/tint.html}{tint: Tint is not Tufte}}. For the time being, the source is LaTeX, not markdown.

\vspace{2\baselineskip}

Markus Lippitz \\ Bayreuth, September 18, 2023



\part{Rays and beams}


\renewcommand{\lastmod}{September 18, 2023}
\renewcommand{\chapterauthors}{Markus Lippitz}

\chapter{Ray optics}

\section{Overview}



%--------------------
\printbibliography[segment=\therefsegment,heading=subbibliography]

\renewcommand{\lastmod}{September 18, 2023}
\renewcommand{\chapterauthors}{Markus Lippitz}

\chapter{Gaussian Beams}


\section{Goals}

By the end of this chapter you should be able to explain the electric field in a Gaussian focus.
You can construct a Gaussian beam 'by hand' for typical lens systems and calculate it using the ABCD law.


\section{Overview}

Hering/Martin Kap. 4.6, Saleh/Teich Kap. 2,  3 und 10, Hecht Kap. 13



%-----------------------------------------------------------------------------
\section{Postulates of Wave Optics}

We expand our model to describe light. In this and the following chapter we use wave optics. We assume that light is a scalar wave. The wave function $u(\br, t)$ is complex-valued and fulfills the wave equation
\begin{equation}
    \nabla^2 u - \frac{1}{c^2} \, \frac{\partial^2 u}{\partial t^2} = 0
\end{equation}
with $c = c_0 /n$ the velocity of light in the medium of refractive index $n$. 
We do not yet assign a physical meaning to the wave function $u(\br, t)$. But since you have seen Maxwell's equations elsewhere, you might think of it as one component of the electric field, for example. At interfaces between media, the index of refraction $n$ changes and thus also $1/c$, but we still do not discuss the physics of such interfaces and partial reflection is beyond our scope.
The only connection we make to observable physical quantities is by defining the  \emph{intensity} $I$ of the wave as
\begin{equation}
    I(\br) = \braket{ |u(\br, t) |^2 }
\end{equation}
where the pointed brackets indicate a time average over a period long compared to the wave period.

A consequence of the linear wave equation is the superposition principle. If $u$ and $v$ are solutions to the wave equation, then also $\alpha u + \beta v$ is a solution. This also means that light beams cross themselves without interaction.


\section{Monochromatic waves}

The solutons of the wave eqaution can be written as harmonic functions
\begin{equation}
    u(\br, t)  = \tilde{u}(\br) \, e^{- i \omega t}
\end{equation}
withj an angular ferqiency $\omega = 2 \pi \nu$. The spatial part $\tilde{u}(\br)$ fulfils the Helmholtz equation
\begin{equation}
    \nabla^2 \tilde{u} + k^2 \tilde{u} = 0 \quad \text{with} \quad k = \frac{\omega}{c}
\end{equation}
$k$ is called the \emph{wavenumber} and becomes the \emph{wavevector} when going to three dimensions. The intensity is then given by $\tilde{u}(\br)$
\begin{equation}
    I(\br) = \braket{ |u(\br, t) |^2 } = |\tilde{u}(\br) |^2 \quad , 
\end{equation}
i.e., the intensity of a monochromatic wave is constant in time. 

Lets discuss a few typical examples

\paragraph*{Plane wave} The amplitude $\tilde{u}$ is given by
\begin{equation}
 \tilde{u}(\br) = A \, e^{i \bk \cdot \br}
\end{equation}
with $\bk$ the wavevector and $|\bk| = k $. The \emph{wavefronts}, i.e., surfaces of constant phase $\phi = q \, 2 \pi =  \arg \tilde{u}(\br)$, are parallel and equidistant planes. The distance is the wavelength $\lambda = c / \nu = 2 \pi / k$.

When the index of refraction $n$ changes at an interface, the frequency $\omega$ remains the same, but the wavelength $\lambda$, the velocity of light $c$ and the wavenumber $k$ change
\begin{equation}
\lambda = \frac{\lambda_0}{n} \qquad
c = \frac{c_0}{n} \qquad
k = \frac{k_0}{n}
\end{equation}
 

\paragraph*{Spherical wave} Here the amplitude $\tilde{u}$ is given by
\begin{equation}
 \tilde{u}(\br) = \frac{A}{r} \, e^{i k r} \quad \text{with} \quad r = |\br|
 \label{eq:2_spherical_wave}
\end{equation}
Note that the right side of the equation does only use scalar variables. The wavefunction depends thus only on the distance to the origin and has spherical symmetry. The wavefronts are concentric spheres of distance $\lambda$.

\paragraph*{Paraboloidal wave} Close to the optical axis, we can approximate the spherical wave by a paraboloidal wave. We call $\theta$
\begin{equation}
    \theta^2 =  \frac{x^2 + y^2}{z^2} \ll 1
\end{equation}
and write $r$ as a Taylor expansion on $\theta$
\begin{align}
    r = & \sqrt{x^2 + y^2 + z^2}= z \sqrt{1 + \theta^2}
    = z \left( 1 + \frac{\theta^2}{2} - \frac{\theta^4}{8} + \cdots \right) \\
    \approx & z \left( 1 + \frac{\theta^2}{2} \right) = z + \frac{x^2 + y^2}{2z} 
\end{align}
This iis called the \emph{Fresnel approximation}. We put it into eq.  \ref{eq:2_spherical_wave} and approximate in the amplitude term even $r \approx z$. We get
\begin{equation}
    \tilde{u}(\br) = \frac{A}{z} \, e^{i k z} \, e^{i k \, \frac{x^2 + y^2}{2z}  }
\end{equation}
For points close to the optical axis but far from the origin, a spherical wave approaches a planar wave. In between, the paraboloidal wave is a useful approximation.


\begin{marginfigure}
    \includegraphics[width=\textwidth]{\currfiledir sketches/paraboloidal.png}
   \caption{XXX sketch Fig 2.2.4 S/T}
\end{marginfigure}


\section{A transparent plate}

As most simple optical element, we consider a transparent plate of thickness $d$ and index of refraction $n$ in air. We transmit a plane wave. The wavefunction is continuos at the interface. We are interested in the complex-valued transmission function $t(x,y)$
\begin{equation}
    t(x,y) = \frac{\tilde{u}(x,y,d)}{\tilde{u}(x,y,0)}
\end{equation}
For perpendicular incidence, the phase advances by $n k_0 d$ from left to right. The transmission function is thus
\begin{equation}
    t(x,y)  = e^{i n k_0 d}
\end{equation}

\begin{marginfigure}
    \includegraphics[width=\textwidth]{\currfiledir sketches/plate.png}
   \caption{A plate}
\end{marginfigure}


When the plane wave approaches the plate under angle $\theta$, then Snell's law gives the internal angle $\theta_i$ as $\sin \theta = n \sin \theta_i$. The wavevector makes this angle $\theta_i$ with the optical axis, so that the $z$-component of the term $\bk \cdot \br$ at the right side gives $n k \cos \theta_i$ and the total transmission function is
\begin{equation}
    t(x,y)  = e^{i n k_0 d \cos \theta_i}
\end{equation}
This is always against my intuition. The geometrical path in the plate gets longer by tilting it, but the phase difference becomes smaller. The point is that we only take the component along $z$ into account, as shifting a plane wave perpendicular to its direction of travel does not change anything.

We of course make again the approximation that the angle $\theta$ is small enough so that we can ignore the $\cos \theta_i$ part.

\begin{marginfigure}
    \includegraphics[width=0.5\textwidth]{\currfiledir sketches/plate_var_d.png}
   \caption{A plate of variable thickness}
\end{marginfigure}


If the plate has a variable thickness $d(x,y)$, we enclose it in a box of thickness $d_0$. Then part of the phase progression goes with $n$, part with air ($n=1$). In total this is
\begin{equation}
    t(x,y) \approx e^{i n k_0 d(x,y)} \,  e^{i k_0 (d_0 - d(x,y))}
    = h_0   e^{i (n-1) k_0 d(x,y)}
\end{equation}
with $h_0 = e^{i  k_0 d_0}$ a constant phase factor. This makes the approximation that all angles are small enough and neighboring parts of the plate do not 'mix' at the output.


\section{Conversion of a plane wave to a spherical wave by a lens}

\begin{marginfigure}
    \includegraphics[width=\textwidth]{\currfiledir sketches/lens.png}
   \caption{A lens as plate of variable thickness}
\end{marginfigure}

The most interesting thin plate of variable thickness is a lens. For simplicity, we use a plane convex lens, i.e, set one radius of curvature to infinity. The thickness $d(x,y)$ of this plate is then
\begin{equation}
    d(x,y) = d_0 - \left( R - \sqrt{R^2 - (x^2 + y^2)} \right)
\end{equation}
We again use the Fresnel approximation $x^2 + y^2 \ll R^2$ and approximate the square-root term
\begin{equation}
    \sqrt{R^2 - (x^2 + y^2)}  = R \sqrt{1- \frac{x^2 + y^2}{R^2}} 
    \approx R \left( 1 -  \frac{x^2 + y^2}{2 R^2} \right)
\end{equation}
so that 
\begin{equation}
    d(x,y) \approx d_0 -  \frac{x^2 + y^2}{2 R^2}
\end{equation}
The transmission function is then
\begin{equation}
    t(x,y)= h_0 \,  e^{-i k_0 \frac{x^2 + y^2}{2f}}
    \quad 
    \text{with} \quad
    f = \frac{R}{n-1}
\end{equation}
and $h_0 = e^{i n k_0 d_0}$ another constant phase factor that we ignore.

A spherical lens thus transforms a plane wave into a paraboloidal wave centered around $z=f$.


\section{Gaussian beams as a paraxial solution of the wave equation}

When we have discussed typical solutions to the wave equation above, we started from the full wave equation, found spherical waves as solution, and then made the paraxial approximation to arrive at the paraboloidal waves. We could also have gone a different route. We can apply the paraxial approximation to the wave equation directly. This leads to the paraxial Helmholtz equation
\begin{equation}
    \nabla_T^2 A + i 2k \frac{\partial A}{\partial z} = 0
    \quad 
    \text{and}
    \quad
    \tilde{u}(\br) = A(\br) \, e^{i k z}
\end{equation}
with $ \nabla_T$ acting only on the transverse coordinates only. The envelop $ A(\br)$ modulates the carrier $\exp(i k z)$. $A$ needs to be \emph{slowly varying}, i.e., on a wavelength length scale it should not change much.

The paraboloidal waves 
\begin{equation}
    \tilde{u}(\br) = \frac{A}{z} \, e^{i k z} \, e^{i k \, \frac{x^2 + y^2}{2z}  }
\end{equation}
i.e.,
\begin{equation}
    A(\br) = \frac{A_1}{z}  \, e^{i k \, \frac{x^2 + y^2}{2z}  }
\end{equation}
fulfil this paraxial Helmholtz equation. The interesting point is that we can come to other solutions of the paraxial Helmholtz equation by replacing $z$ by $q(z) = z - i z_0$, i.e.
\begin{equation}
    A(\br) = \frac{A_1}{q(z)}  \, e^{i k \, \frac{x^2 + y^2}{2q(z)}  }
\end{equation}
These are \emph{Gaussian beams}. We call $q$ the q-parameter and $z_0$ the \emph{Rayleigh range}. We separate the complex function $1/q(z)$ into its real and imaginary part
\begin{equation}
    \frac{1}{q(z)} = \frac{1}{z -i z_0} = \frac{1}{R(z)} + i \frac{\lambda}{\pi W^2(z)}
\end{equation}
We will see that $R$ and $W$ give the wavefront radius of curvature and the beam width, respectively. Putting everything together, the wavefunction reads
\begin{equation}
    \tilde{u}(\br) = A_0 \, \frac{W_0}{W(z)} \, 
    \exp \left( - \frac{\rho^2}{W^2(z)}  \right) \, 
    \exp \left( +i kz +ik  \frac{\rho^2}{2 R(z)}  - i \zeta(z) \right) 
\end{equation}
with
\begin{align}
    W(z) = & W_0 \sqrt{1 + \left( \frac{z}{z_0} \right)^2    } \\
    R(z) = & z \left[ 1 + \left( \frac{z_0}{z} \right)^2 \right] \\
    \zeta(z) = & \arctan \frac{z}{z_0} \\
    W_0 = & \sqrt{\frac{\lambda z_0}{\pi}}
\end{align}
Note that there are only two independent parameters next to the wavelength $\lambda$, namely the amplitude $A_0$ and the Rayleigh range $z_0$.

\section{Gaussian beams as eigenmodes of a resonator}
The importance of Gaussian beams comes from the laser as a ubiquitous light source. A laser produces Gaussian beams because these wave functions are the eigenmodes of a resonator formed by two spherical mirrors.

In a laser, we are interested in eigenmodes, i.e. optical wave functions that do not change as they bounce back and forth in the resonator. The mirrors in a laser cavity are typically so highly reflective that there are many round trips before the field leaves the cavity.


\begin{marginfigure}
    \includegraphics[width=\textwidth]{\currfiledir sketches/cavity.png}
   \caption{Eigenmodes of a laser cavity}
\end{marginfigure}


For an eigenmode to occur, the wavefront of the mode at the position of the mirror must match the shape of the mirror, otherwise it will reflect back into itself. The design of the cavity gives the radius of curvature $R_1$ and $R_2$ and the distance $d$ between the mirrors. We now show that under certain conditions a Gaussian beam is an eigenmode of such a cavity.

We search for the positions $z_1$ and $z_2$ of the mirrors and the Rayleigh range $z_0$ if the mean. We have the equation system
\begin{align}
    z_2 = & z_1 + d \\
    R_1 = & z_1 \left[ 1 + \left( \frac{z_0}{z_1} \right)^2 \right] \\
    R_2 = & z_2 \left[ 1 + \left( \frac{z_0}{z_2} \right)^2 \right] \\
\end{align}

The solution is
\begin{align}
    z_1 = & \frac{-d (R_2 + d)}{R_1 + R_2 + 2d} \\
    z_2 = &z_1 + d  \\
    z_0^2 = & \frac{-d (R_1 +d)(R_2 + d)(R_1 + R_2 +d)}{(R_1 + R_2 + 2d)^2}
\end{align}
For a Gaussian beam $z_0$ must be real (or $z_0^2 > 0$). Otherwise $q = z - i z_0$ would be real and we would get a paraboloidal wave. This results in the \emph{stability condition} of a spherical cavity
\begin{equation}
    0 \le \left( 1 + \frac{d}{R_1} \right) \left( 1 + \frac{d}{R_2} \right) \le 1
\end{equation}

XXX sketch

\section{Parameters and Properties of Gaussian Beams}

\section{Degrees of freedom of a Gaussian beam}

\section{ABCD Law}



\section{Bonus: Hermit and Laguerre Gaussian beams}


\section{Technique: Knife Edge Test}





%--------------------
\printbibliography[segment=\therefsegment,heading=subbibliography]


\part{Fourier optics}
\include{3_Fourier_optics/3_fourier_optics}

\part{Light in matter}
\renewcommand{\lastmod}{November 24, 2023}
\renewcommand{\chapterauthors}{Markus Lippitz}

\chapter{Dielectric Materials}
\label{chap:dielectrics}

\goal{By the end of this chapter you should be able to explain and experimentally demonstrate total internal reflection and the Brewster effect.}



\section{Overview}

With this chapter we begin to consider the optical properties of media beyond their refractive index. The physics of the medium will play a role and have consequences for the propagation of light. To be able to do this, we also have to describe light as an electromagnetic wave, with three components for the electric and magnetic field, and not only as a scalar wave as in the last chapters. This will lead to the phenomenon of absorption and dispersion. We will also be able to assign a value to the amplitude of the reflected and transmitted waves at an interface.  These topics are described in chapter 5 and 6 of \cite{SalehTeich1991} and chapter 3 of \cite{Hecht_Optics}.


\section{Maxwells equations}

For completeness, let us start with the Maxwell equations in their macroscopic form
\begin{align}
\nabla \, \bD = & \rho \\
\nabla \, \bB = & 0 \\
\nabla \times \bE =  & - \dot{\bB} \\
\nabla \times \bH = & \dot{\bD} + \bj \quad .
\end{align} 
Matter comes in by the respective material equations
\begin{align}
    \bD = & \epsilon_0 \bE +  \bP = \epsilon \epsilon_0 \bE \\
    \bH = & \frac{1}{\mu_0} \bB - \bM =   \frac{1}{\mu \mu_0} \bB \\
    \bj = & \sigma \bE \quad .
\end{align}
Note that I use a unit-free dielectric function $\epsilon$. In literature, one finds different other methods to write the term $\epsilon \epsilon_0$. At the second equal sign we have assumed in each case a linear and isotropic medium. Let us define these and similar terms:
%
\begin{description}
    \item[linear] The relation between the electric field $\bE(\br, t)$ and the polarization $\bP(\br, t)$ is linear.
    \item[isotropic] The relation between $\bE$ and $\bP$ is independent of the direction of $\bE$. This also means that $\bE$ and $\bP$ are parallel.
    \item[homogeneous] The relation between $\bE$ and $\bP$ is independent of the position $\br$.
    \item[nondispersive] The relation between $\bE$ and $\bP$ is instantaneous, i.e., it depends only on the value of $\bE$ at time $t$, but not on earlier times. As we will see, this is equivalent to saying that the relation does not depend on the frequency $\omega$ of light. This is a thought model and is only approximated by real materials.
    \item[local] The relation between $\bE$ and $\bP$ depends only on the value of $\bE$ at one point $\br$, not at other points. This is also called \emph{spatially nondispersive}. Optical active media (next chapter) are nonlocal.
\end{description}


\section{Wave equations}
When we assume a source-free medium ($\bj = 0$, $\rho = 0$), one can derive the wave equation for an isotropic  and linear medium
\begin{equation}
    \nabla^2 \, \bE = \frac{n^2}{c_0^2} \ddot{\bE} \quad \text{with} \quad c_0^2 = \frac{1}{\mu_0 \epsilon_0} \quad ; \quad   n^2 = \epsilon \quad; \quad  \mu \approx 1 
\end{equation}
A similar equation exists for $\bH$. The individual vector components of the electrical and magnetic field fulfil thus the scalar wave equation of chapter \ref{chap:2_gauss}.

The flow of electromagnetic energy is described by the Poynting vector\sidenote{John Henry Poynting, 1852--1914}
\begin{equation}
    \bS = \bE \times \bH
\end{equation}
The intensity $I$ of a wave on a surface with normal $\boldsymbol{n}$ is the temporal average of the Poynting vector, i.e.
\begin{equation}
    I = \braket{\bS \cdot \boldsymbol{n}}_T = \frac{c \, n \, \epsilon_0}{2} \, | \bE_0 |^2 = \frac{1}{2 \eta } \, | \bE_0 |^2
\end{equation}
where $\bE_0 $ is the amplitude of the electrical field and $\eta = \sqrt{ \mu \mu_0 / (\epsilon \epsilon_0)}$ the impedance of the medium. For vacuum, $\eta_0 \approx$~\SI{377}{\ohm} . An intensity of 10~W/cm$^2$ corresponds to an electric field of about 87~V/m.

The Poynting vector fulfills the Poynting theorem: the flow of energy through a surface enclosing a volume either changes the energy density within that volume or performs work on magnetic or electric dipoles. As equation:
\begin{equation}
    \nabla \bS = - \frac{\partial}{\partial t} \left( \frac{1}{2}  \epsilon \epsilon_0 \bE^2 + \frac{1}{2} \mu \mu_0 \bH^2  \right)
    + \bE \cdot \frac{\partial \bP}{\partial t} +   \mu_0 \bH \cdot \frac{\partial \bM}{\partial t}
\end{equation}


As with scalar waves, we find different solutions to the wave equation. The plain wave also exists as electromagnetic wave:
\begin{align}
    \bH(\br, t) = & \bH_0 \, e^{i (\bk \br - \omega t)} \\
    \bE(\br, t) = & \bE_0 \, e^{i (\bk \br - \omega t)} 
\end{align}
with $|\bk| = k = 2 \pi n / \lambda_0$ and $\bH_0$, $\bB_0$ and $\bk$  orthogonal on each other. The electromagnetic wave is thus a \emph{transversal} wave.

The vectorial electromagnetic forms of paraboloidal wave and Gaussian beams can be constructed by vectorizing the scalar waves $u(\br)$ of the preceding chapters:
\begin{equation}
    \bE(\br) = \mathcal{E}_0 \left( - \hat{\bx} + \frac{x}{z + i z_0} \hat{\bz} \right) u(\br)
\end{equation}
where $\hat{\bx}$ and $\hat{\by}$ are unit vectors pointing in x and z direction, respectively, and $\mathcal{E}_0 $ is a scalar amplitude. $z_0$ is set to zero for a paraboloidal wave.

\section{Phenomenological approach to absorption}

Let us begin by describing absorption in media without attributing a microscopic origin. The susceptibility $\chi$ is complex-valued
%\sidenote{note the minus in front of the $i$!}
, i.e. $\chi = \chi' + i \chi''$ and thus the dielectric function 
\begin{equation}
    \epsilon = 1 + \chi = 1 + \chi' + i \chi'' = \epsilon' + i \epsilon''
\end{equation}
This means that the wave number $k$ will become complex, too
\begin{equation}
    k = \frac{\omega}{c} =  k_0 \sqrt{\epsilon} = k_0 \sqrt{1 + \chi' + i \chi''} = \beta + i \frac{\alpha}{2}
\end{equation}
The meaning of the real-valued $\alpha$ and $\beta$ will become clear when we use this definition in a plane wave:
\begin{equation}
    \mathcal{E}(z,t) =  \mathcal{E}_0 \,  e^{ i (k z - \omega t)} =
     \mathcal{E}_0 \, e^{-i \omega t} \, e^{i \beta z} \, e^{- \alpha z / 2}
\end{equation}
The intensity of this waves thus drops as
\begin{equation}
    I(z) \propto |\mathcal{E}(z,t) |^2 = |\mathcal{E}_0|^2 \, e^{- \alpha z}
\end{equation}
$\alpha$ is thus the absorption coefficient\sidenote{or attenuation or extinction coefficient}. Positive $\alpha$ means a decay of intensity, negative $\alpha$ would mean a gain, as in a laser. $\beta$ describes the progression of the phase or wave fronts. It is related to the real part $n$ of the refractive index\sidenote{I use the form $\tilde{n}= n + i \kappa$.} by $\beta = n k_0$. Everything together we have
\begin{equation}
    n + i \kappa = \frac{\beta}{k_0} + i \, \frac{1}{2} \, \frac{\alpha}{k_0} = \pm \sqrt{1 + \chi' + i \chi''}
    \label{eq:4_n_chi}
\end{equation}
The sign of the square root is chosen such that a  positive (absorbing) $\chi''$ leads to a positive (absorbing) $\alpha$, independent of the sign of $\chi'$. As we will see below $\chi' < 0$ is possible, e.g., near resonances.

It is convenient to have approximate forms of eq.  \ref{eq:4_n_chi} for the limiting cases of weak and strong absorption
\begin{align}
    \chi'' \ll 1 + \chi' &  \rightarrow& n \approx \sqrt{1 + \chi'} && \alpha \approx  \frac{k_0}{n} \chi'' \\
    \chi'' \gg |1 + \chi'| & \rightarrow & n \approx \sqrt{\chi'' / 2} && \alpha \approx 2 k_0 \sqrt{ \chi'' / 2}
\end{align}



\section{The Kramers-Kronig relations}

So far, we have only discussed the relationship between the applied external field $E(t)$ and the resulting polarization $P(t)$ for 'monochromatic' fields of the type $\exp(-i \omega t)$, i.e. for a precisely defined frequency $\omega$:
\begin{equation}
P(t) = \chi(\omega) \epsilon_0 E(t) \quad \text{for} \quad E(t) = E_0 e^{-i \omega t} \quad .
\end{equation}
This gave the frequency dependence of $\chi(\omega)$. We can generalize this for any time evolution of the field $E(t)$. The susceptibility is the \emph{impulse response} of the material, the memory so to speak:
\begin{equation}
P(t) = \epsilon_0 \int_{-\infty}^{+\infty} \chi( \Delta t = t - t') \, E(t') \, dt' \quad \text{for} \quad E(t) = \text{any} \quad .
\end{equation}
The polarization $P$ now, i.e. at time $t$, depends on the electric field at all other times $t'$. How strong the fields are depends only on the time interval $\Delta t$. Causality requires that the polarization 'now' does not depend on the field amplitudes in the future. Therefore $\chi( \Delta t = t - t' < 0) $ must be zero. This means that the susceptibility $\chi( \Delta t ) $ is complex, but known over half of the time ray as fixed to zero. This has consequences for the Fourier transform, i.e. for $\chi(\omega)$.

These consequences can be derived with the help of function theory\sidenote{see also Appendix A of \cite{Yariv1989}} and are the Kramers-Kronig relations. The following relationship exists between the real ($\chi'$) and imaginary ($\chi''$) parts of the susceptibility if they obey causality:
\begin{align}
 \chi'(\nu) = & \frac{2}{\pi} \, P \int_0^\infty \frac{s \, \chi''(s)}{s^2 - \nu^2} \, ds \\
 \chi''(\nu) = & \frac{2}{\pi}\, P \int_0^\infty \frac{\nu \, \chi'(s)}{\nu^2 - s^2} \, ds \quad .
 \label{eq:diel_KK}
\end{align}
$P$ denotes the Cauchy principal value integral. Similar relationships also exist for $\chi(\omega)$ and $\epsilon(\omega)$ as well as for all other variables that are subject to causality.

In principle, it is therefore sufficient to measure the real part of the susceptibility $\chi(\omega)$ in order to determine the imaginary part and thus the complete complex-valued function. Unfortunately, however, the integrals in Eq~\ref{eq:diel_KK} run over the entire frequency range from zero to infinity, which is of course not accessible experimentally. The Kramers-Kronig relations can still be used sensibly by making appropriate assumptions about the course outside the measured interval.



\section{Lorentz oscillator model}

The response of matter to an electric field is governed by the charged ions and electrons. Restoring forces lead to resonances depending on the frequency of the optical field. In the infrared, bound ions resonate, while in the ultraviolet, bound electrons dominate.


\begin{marginfigure}
\inputtikz{\currfiledir lorentz_oscillator}

\caption{Frequency dependence of the real and imaginary parts of the Lorentz oscillator. The real and imaginary parts of the complex-valued refractive index $\tilde{n}$ look qualitatively the same. \label{fig:4_lorentz}}
\end{marginfigure}

The Lorentz oscillator model is a simple model that can be used to describe the frequency dependence of the dielectric function in the vicinity of resonances. In a damped harmonic oscillator (mass $m$, damping constant $\gamma$, natural frequency $\omega_0$), the mass is deflected by a periodic electric field (amplitude $E_0$, frequency $\omega$) by $x$ because the mass carries a charge $e$. All together
\begin{equation}
 m \ddot{x} + \gamma \dot{x} + m \omega_0^2 x = e E_0 e^{- i \omega t} \quad .
\end{equation}
The stationary solution of this differential equation is
\begin{equation}
 x(t) = \frac{e \, E_0}{m (\omega_0^2 - \omega^2) - i \gamma \omega} \, e^{- i \omega t} \quad .
\end{equation}
The macroscopic polarization $P$ is the sum of all microscopic polarizations, i.e.
\begin{equation}
P(t) = N \, e \,x(t) = (\epsilon -1 ) \epsilon_0 \, E_0 e^{- i \omega t}
= \chi \epsilon_0 E(t) \quad .
\end{equation}
This results in the dielectric function
\begin{equation}
\epsilon(\omega) = 1 + N \alpha = 1 +\frac{N e^2}{\epsilon_0} \frac{1}{m (\omega_0^2 - \omega^2) - i \gamma \omega} = \epsilon' + i \epsilon'' \quad .
\end{equation}
Explicit real and imaginary parts are
\begin{align}
 \epsilon' = & 1 + \frac{N e^2}{\epsilon_0} \frac{ m (\omega_0^2 - \omega^2)}{m^2 (\omega_0^2 - \omega^2)^2 + \gamma^2 \omega^2}  \\
  \epsilon'' = & \frac{N e^2}{\epsilon_0} \frac{ \gamma \omega }{m^2 (\omega_0^2 - \omega^2)^2 + \gamma^2 \omega^2}  \quad .
\end{align}


\begin{questions} 
\item Analogous to Figure \ref{fig:4_lorentz}, show the frequency dependence of the components of the refractive index, i.e. of $n$ and $k$.

\item Approximate the real and imaginary parts of $\epsilon$ near resonance at $\omega_0$ as a function of $\Delta \omega = \omega - \omega_0$. In the case of the real part, only the range $| \Delta \omega | \gg \gamma/m$ is of interest.
\end{questions}




\section{Normal and  anomalous dispersion}

The visible spectral region is at a higher frequency than the resonance of the bound ions in the infrared, but at a lower frequency than that of the bound electrons in the ultraviolet. The real part $n$ of the refractive index   increases  with frequency, i.e. $n(\text{blue}) > n(\text{red})$ (see Fig. 
\ref{fig:4_multiple_lorentz}). This is called 'normal' dispersion. It causes red light to deviate less than blue light in a prism and to be focused by a lens at a greater distance. On the energetically 'other' side of a resonance, the opposite behavior can be observed, 'anomalous dispersion'.

\begin{marginfigure}
    \inputtikz{\currfiledir multiple_lorentz_oscillator}
\caption{The visible spectral range lies between two resonances. \label{fig:4_multiple_lorentz}}
\end{marginfigure}


The Lorentz-shaped resonance can be shown in a demonstration experiment. The imaginary part of the dielectric function Fig.~\ref{fig:4_lorentz} determines the absorption and thus the line shape in the absorption spectrum of atoms or molecules. The real part determines the dispersion, i.e. the refractive index of a medium. A simple method of determining the refractive index is to use a prism made of the material to be examined. In a prism, the deflection of the light beam is proportional to the difference of the refractive index inside compared to outside (actually always air $\approx$ vacuum). However, the electronic resonance  must also be shifted from the ultraviolet to the visible. In the experiment, a prism made of sodium vapor is used for this purpose. The strong absorption of the sodium D lines at a wavelength of around 589~nm produces a highly visible effect. 


\begin{marginfigure}
\includegraphics[width=\textwidth]{\currfiledir dispersion.png}
\caption{Anomalous dispersion in sodium vapor. }
\end{marginfigure}


Sodium vapor is generated in an evacuated tube by strongly heating solid sodium.  The tube is heated from below and cooled from above so that the vapor density decreases towards the top. This corresponds to a prism with its tip pointing upwards. Here, too, the effective glass thickness decreases towards the top when averaged over the entire beam path. The light beam is then passed through a glass prism with a vertical axis to create a horizontal wavelength axis. The result is a spectrum as shown in the adjacent figure. The horizontal axis is proportional to the wavelength, the vertical axis to the deviation of the refractive index from unity. The spectrum is interrupted at the absorption line itself because the sodium vapor completely absorbs the light there. It can be seen that the refractive index falls below unity at the higher energy side of the resonance.


\section{Reflection and transmission}

Now that we can describe matter, we want to know how much of a wave is transmitted through an interface between two media and how much is reflected. Let us assume that the ray travels in the xz-plane. The surface is an xy plane. As we will see in the next chapter on polarization optics, it is sufficient to examine the response for linearly polarized light, where the direction of polarization is either in the plane defined by the rays (xz) or perpendicular to it (y). The first case is called p-polarized (p for parallel) or transverse magnetic (TM), since the magnetic field is orthogonal to the xz-plane of incidence. The second case is called s-polarized (s as 'senkrecht', perpendicular) or transverse electric (TE) because the electric field is perpendicular to the xz-plane of incidence.

At the interface, the sum of incident and reflected wave on each side has to match the transmitted wave on the other side. Matching means that the phases need to agree and the amplitudes need to follow the continuity conditions of electromagnetic fields. The phase argument defines the angles of reflected and transmitted waves, luckily identical to geometrical optics. The continuity argument defines the amplitudes as reflection $r_{12}$ and transmission $t_{12}$ coefficients for the electric field. These are called the \emph{Fresnel equations}. This calculation can be found in textbooks on electrodynamics, e.g, \cite{Nolting-ED}.


We follow here \cite{Novotny-Hecht2012}, who follow \cite{BornWolf2002}, especially in the direction of the field vectors, see Fig. 2.2 in \cite{Novotny-Hecht2012}. In this definition,  $r^s$ and $r^p$ differ at normal incidence by a factor of $-1$. We assume non-magnetic materials ($\mu = 1$) and describe the propagation direction by the z component $k_z$ of the wave vector $\bk$. For a wave traveling from medium 1 towards medium 2 we get
\begin{align}
 r_{12}^s = & \frac{k_{z,1} - k_{z,2}}{k_{z,1} + k_{z,2}}  = - r_{21}^s\\
 t_{12}^s = & \frac{2 \, k_{z,1}}{k_{z,1} + k_{z,2}} =  \frac{k_{z,1}}{k_{z,2}}  \,  t_{21}^s\\
  r_{12}^p = & \frac{\epsilon_2	 k_{z,1} - \epsilon_1 k_{z,2}}
				  {\epsilon_2 k_{z,1} + \epsilon_1 k_{z,2}}  = - r_{21}^p\\
  t_{12}^p = & \frac{2 \sqrt{\epsilon_1 \epsilon_2}	 \,k_{z,1} }
				  {\epsilon_2 k_{z,1} + \epsilon_1 k_{z,2}}  = \frac{k_{z,1}}{k_{z,2}}  \,  t_{21}^p \quad . 
\end{align}
We could also write these coefficients in terms of angle of incidence $\theta$ with
\begin{equation}
 \theta = \arcsin \frac{k_x}{n k_0} = \arcsin \sqrt{1 - \left( \frac{k_z}{n k_0} \right)^2 } \quad . 
\end{equation}
This would also hold in the case of evanescent waves ($k_x > n k_0$) when we allow complex angles $\theta$. We nowhere need that $\theta$ is a geometrical angle. We only need that $n \sin \theta$ is the same on both sides.

\begin{marginfigure}
\inputtikz{\currfiledir Fresnel_field_external}

\inputtikz{\currfiledir Fresnel_field_internal}
\caption{Fresnel coefficients  $r = |r|e^{i \phi}$ for external (top) and internal (bottom) reflection at an air--glass interface. red: s-polarized, blue: p-polarized. dashed: phase \label{fig:4_Fresnel_field}}
\end{marginfigure}

Figure \ref{fig:4_Fresnel_field} shows the amplitude and phase of the reflection coefficient $r$ for a reflection at an air--glass and a glass--air interface. Coming from the less dense medium, we find a zero reflectivity for p-polarized light. This is the Brewster effect: the incident field induces oscillating dipoles at the surface of the medium, aligned with the polarization direction of the field. A dipole emits radiation in all directions, but not in the direction of its oscillation. If this direction of oscillation is in the expected direction of the outgoing wave, the amplitude of that wave must be zero.

When light is incident from the dense medium, we observe total internal reflection for both polarization directions above a critical angle. As we have already seen with Fourier optics, in these cases the in-plane component of the wave vector on the glass side is larger than the total length of the wave vector on the air side. So we have evanescent waves on the air side. Note that even though the reflectivity $|r|$ is always one above the critical angle, the reflected field acquires a phase that depends on the angle of incidence and the direction of polarization.


These coefficients are for the fields. The reflected power is the fraction $R = |r|^2$ of the incident power. The transmitted power is the fraction $T$ with
\begin{equation}
    T = 1 - R \neq |t|^2 \quad .
\end{equation}
This inequality is caused by the differing impedance on both sides of the interface and the differing direction of travel due tue refraction. The orientation of the 'power meter surface' needs to change. Both can be corrected so that we get for a wave traveling from 1 to 2
\begin{equation}
    T = \frac{n_2 \cos \theta_2}{n_1 \cos \theta_1} \, |t|^2 \quad .
\end{equation}




\begin{marginfigure}
\inputtikz{\currfiledir Fresnel_power_external}

\inputtikz{\currfiledir Fresnel_power_internal}

\caption{Reflected (thick) and transmitted (thin) power for external (top) and internal (bottom) reflection at an air--glass interface. red: s-polarized, blue: p-polarized \label{fig:4_Fresnel_power}}

\end{marginfigure}




%--------------------
\printbibliography[segment=\therefsegment,heading=subbibliography]
 
\renewcommand{\lastmod}{December 7, 2023}
\renewcommand{\chapterauthors}{Markus Lippitz}

\chapter{Polarization and anisotropic media}



\goal{By the end of this chapter you should be able to explain and experimentally demonstrate the operation of a wave plate.}


\section{Overview}

Polarization is the most important property of light. It makes the difference to scalar waves. It gives an additional degree of freedom in optical devices. We first introduce the Jones formalism to describe the state of polarization and then discuss the propagation of light in anisotropic media. The most surprising effect is probably birefringence. The technical applications are components of polarization optics such as waveplates and polarizing prisms. Spectroscopic applications include the magneto-optical Kerr effect and circular dichroism.


\section{Introduction}

\begin{marginfigure}
    \includegraphics[width=40mm]{\currfiledir sketches/pol_field.png}
    \caption{Electric field in the $xy$-plane.}
\end{marginfigure}

We start by looking at a monochromatic plane wave again. This time, however, we pay more attention to the vectorial amplitude by writing
\begin{equation}
    \bE(z,t) = \Re \left\{ 
    \left( A_x \hat{\bx} + A_y \hat{\by} \right) \, e^{i \omega (t - z/c)}
    \right\}
\end{equation}
where $A_{x,y}$ are complex-valued amplitudes. We separate them intro amplitude $a$ and phase $\phi$ (and write down only the $x$ components for simplicity)
\begin{equation}
    A_x = a_x e^{i \phi_x}
\end{equation}
so that the $x$ components of the field is
\begin{equation}
    E_x = a_x \cos \left( \omega (t - z/c) + \phi_x \right) \quad .
\end{equation}
In the $xy$-plane this describes as function of time $t$ an ellipse with ellipticity $\chi$ and orientation $\psi$:
\begin{align}
    \tan 2 \psi = & \frac{2 R}{1 - R^2} \cos (\phi_y - \phi_x) \\
    \sin 2 \psi = & \frac{2 R}{1 - R^2} \sin (\phi_y - \phi_x) 
\end{align}
with $R = a_y / a_x$.

\begin{questions}
    \item Draw the field $\bE$ for a fixed $t$ in the $xyz$-cube and for a fixed $z$ in the $xyt$-cube.
\end{questions}


\section{Polarization states}

\emph{Linear polarization} is described by
\begin{equation}
    \phi_y - \phi_y = 0 \text{ or } \pi
\end{equation}
In these cases, the ellipticity vanishes $\chi = 0$ and the orientation $\psi$ is 
\begin{equation}
    \tan 2 \psi = \frac{2R}{1 - R^2} \quad \text{or} \quad \tan \psi = \frac{a_y}{a_x} \quad .
\end{equation} 
%
\emph{Circular polarization} is described by
\begin{equation}
    \phi_y - \phi_y = \pm \frac{\pi}{2} \quad \text{and} \quad a_x = a_y
\end{equation}
The electric field vector describes a circle in the $xy$-plane. The plus sign corresponds to right-circularly polarized (RCP), the minus sign to left-circularly polarized (LCP). A snapshot in time gives in the $xyz$ cube a helix of the corresponding\sidenote{This sign differs in some books!} handedness.

All other polarization states can be depicted as point on the \emph{Poincaré sphere}. We need only two parameters to describe the state of polarization, $\chi$ and $\psi$, or $R$ and $\phi_y - \phi_x$. For the Poincaré sphere, we set 
\begin{equation}
    r = 1 \qquad \theta = \frac{\pi}{2} - 2 \chi \qquad \psi = 2 \chi
\end{equation}
North and south pole correspond to RCP and LCP, respectively. The equator describes linearly polarized light ($\chi =  0$) in the different directions.

\begin{marginfigure}
    \includegraphics[width=50mm]{\currfiledir sketches/poincare.png}
    \caption{Polarization states on the Poincaré sphere.}
\end{marginfigure}

A more general method to describe polarized light is the four-component \emph{Stokes vector}. The first component is the intensity of the light field that was not contained in the Poincaré sphere. The other three are the three cartesian components of the point on the Poincaré sphere, i.e.
\begin{align}
    S_0 = & a_x^2 + a_y^2 \\
    S_1 = & a_x^2 - a_y^2 \\
    S_2 = & 2 a_x  a_y \cos (\phi_y - \phi_x )\\
    S_3 = & 2 a_x  a_y \sin (\phi_y - \phi_x )
\end{align}
As we have introduced the $a_{x,y}$ above, we always get $S_1^2 + S_2^2 + S_3^2 = 1$, which would not justify an additional component. However, the scheme can be extended to describe partially polarized or unpolarized light, for which the last component is needed.



\section{Jones formalism}

The simplest way to describe the effect of optical elements such as quarter and half wave plates or polarizers on the polarization state of light is the Jones formalism. We write the complex amplitudes $A_x$ and $A_y$ as the two components of a Jones vector
\begin{equation}
    \bJ = \begin{pmatrix}
        A_x \\ A_y
    \end{pmatrix}
\end{equation}
In most cases we normalize $|\bJ| = 1$. Examples are 
\begin{equation}
    \begin{pmatrix}
       1 \\ 0
    \end{pmatrix} 
    \qquad 
      \begin{pmatrix}
        \cos \theta \\ \sin \theta
     \end{pmatrix}
     \qquad
     \frac{1}{\sqrt{2}}
     \begin{pmatrix}
        1 \\ i
     \end{pmatrix} 
     \qquad 
     \frac{1}{\sqrt{2}}
     \begin{pmatrix}
        1 \\ -i
     \end{pmatrix} 
\end{equation}
for linear polarization in x direction, under an angle $\theta$ to the x-direction, and right and left-handed circular polarization.

A $2 \times 2$ matrix describes the effect of an optical element. A \emph{linear polarizer} is
\begin{equation}
    T = 
    \begin{pmatrix}
   1 & 0 \\ 0 & 0
    \end{pmatrix} 
\end{equation}
A \emph{wave plate} delays the slow y component by a phase $\phi$ compared to the fast x component, i.e.
\begin{equation}
    T = 
    \begin{pmatrix}
   1 & 0 \\ 0 & e^{-i \phi}
    \end{pmatrix} 
\end{equation}
As light travels faster when polarized along the x axis, it is called the 'fast axis' of the wave plate.
Important cases are the quarter-wave plate with $\phi = \pi/2 = 2 \pi / 4$ and the half-wave plate with  $\phi = \pi = 2 \pi / 2$. A \emph{quarter wave plate} converts linear polarized light under $45^\circ $ into LCP, and RCP in linear polarized light under $45^\circ$:
\begin{equation}
    \begin{pmatrix}
        1 \\ -i
     \end{pmatrix}
     = 
    \begin{pmatrix}
        1 & 0 \\ 0 & e^{-i \pi/2}
         \end{pmatrix} \cdot
         \begin{pmatrix}
            1 \\ 1
         \end{pmatrix}
         \quad \text{and} \quad
         \begin{pmatrix}
            1 \\ 1
         \end{pmatrix}
         = 
        \begin{pmatrix}
            1 & 0 \\ 0 & e^{-i \pi/2}
             \end{pmatrix} \cdot
             \begin{pmatrix}
                1 \\ i
             \end{pmatrix}
\end{equation}
A \emph{half wave plate converts} turns the polarization direction of
linear polarized light from $+45^\circ $ to $-45^\circ $, i.e., by $90^\circ$. It also converts RCP into LCP:
\begin{equation}
            \begin{pmatrix}
                1 \\ -1
             \end{pmatrix}
             = 
            \begin{pmatrix}
                1 & 0 \\ 0 & e^{-i \pi}
                 \end{pmatrix} \cdot
                 \begin{pmatrix}
                    1 \\ 1
                 \end{pmatrix}
                 \quad \text{and} \quad
                 \begin{pmatrix}
                    1 \\ -i
                 \end{pmatrix}
                 = 
                \begin{pmatrix}
                    1 & 0 \\ 0 & e^{-i \pi}
                     \end{pmatrix} \cdot
                     \begin{pmatrix}
                        1 \\ i
                     \end{pmatrix}
\end{equation}
It is important to note that these conversions only work for the indicated relative orientation of the waves plates x and y axis and the incoming polarization state. In almost all cases, one needs to rotate the  optical elements relative to the lab-frame $xy$ coordinate system. The effect of a rotated element $T'(\theta)$ is obtained by rotation matrices
\begin{equation}
    T'(\theta) = R(\theta) \cdot T \cdot R(-\theta) \quad \text{with} \quad
    R(\theta) = 
    \begin{pmatrix}
        \cos \theta & \sin \theta \\ 
        - \sin \theta & \cos \theta 
    \end{pmatrix}
\end{equation}

\begin{questions}
    \item Find orientations of quarter- and half-wave plate that do not change a linearly polarized beam!
    \item Plot the total power in a beam as function of angle $\theta$ of a linear polarizer relative to a linear polarized beam and to a left-circular polarized beam.
\end{questions}


\section{Anisotropic media} 

The refractive index of anisotropic media depends on the polarization direction of the light wave. This statement defines anisotropic media. This relationship makes anisotropic media of special interest for polarization optics, and polarization optics of importance for the study of anisotropic media. Before we get to the propagation of light in anisotropic media, let us discuss the microscopic patterns of  anisotropy. On the nanoscale, molecules could be ordered in position and / or orientation.
\begin{description}
    \item[gases, liquids, amorphous solids] Both position and orientation are random. The media are isotropic
    \item[polycrystallline solids] On a short length scale we find order in position and orientation, leading to anisotropy. On a longer length scale, the crystallites are disordered, averaging out the anisotropy. 
    \item[crystalls] Both position and direction are ordered. In general case, crystals are anisotropic. 
    \item[liquid crystals] The position is random, but the orientation is ordered. This is enough to find anisotropy. 
\end{description}

\begin{marginfigure}
    \includegraphics[width=30mm]{\currfiledir sketches/order.png}
    \caption{Order in position and / or orientation.}
\end{marginfigure}



\section{Index ellipsoid }

When we deal with anisotropic media, we have to take the tensorial nature of the electric permittivity $\beps$ into account:
\begin{equation}
   \bD = \epsilon_0 \beps \bE \quad \text{or} \quad  D_i = \epsilon_0 \sum_j \epsilon_{i j} \, E_j \quad \text{with} \quad i,j = x, y, z  \label{eq:5_eps_trensor}
\end{equation}
where $\epsilon_{i j} $ are the components of the $3\times 3$ tensor $\beps$. For most materials, the tensor is symmetric, i.e., $\epsilon_{ij} = \epsilon_{ji}$.  This holds for nonmagnetic dielectrics that do not show optical activity, and in absence of magnetic fields. The symmetry reduces the originally 9 elements to 6 independent ones.\sidenote{see chapter 15.4 of \cite{Brooker_Optics} for an interpretation of the coefficients}

We can depict such a symmetric tensor as ellipsoidal surface defined by 
\begin{equation}
     \sum_{i,j} \epsilon_{i j}  x_i x_j = 1
\end{equation}
When we change the coordinate system, then both the $x_i$ and the $\epsilon_{ij}$ change and the ellipsoid remains unchanged. There is one coordinate system in which the $\epsilon_{ij}$ matrix is diagonal. This are the principal axes.  In these directions, $\bE$ amd $nD$ are parallel. In the following, my spatial coordinate system is always this principal system, labeled as $x_1$, $x_2$, $x_3$ The semi-axes of the ellipsoid, i.e., where the ellipsoid crosses the axes, have the values $1/\sqrt{\epsilon_i} = 1/n_i$. Here $n_i$ is the index of refraction along one of the principal axes.


\begin{marginfigure}
    \inputtikz{\currfiledir ellipsoid}
    \caption{The index ellipsoid for an uniaxial crystal.}
\end{marginfigure}


In the following, it is more useful to have the inverse tensor $\boldsymbol{\eta} = \beps^{-1}$, i.e.
\begin{equation}
    \bE = \frac{1}{\epsilon_0} \beps^{-1} \bD  = \frac{\boldsymbol{\eta}}{\epsilon_0}  \bD
 \end{equation}
When depicted as ellipsoid, called \emph{index ellipsoid}, its principal axes agree with the permittivity tensor. The intersections with the axes are at $n_i$ and we can write
\begin{equation}
    \sum_{i=1,2,3} \frac{x_i^2}{n_i^2} = 1
\end{equation}



We distinguish crystals by their set of $n_i$:
\begin{align}
          &  n_1 = n_2 = n_3             & & \quad \text{isotropic} \\
 n_{o} = &  n_1 = n_2  \neq n_3  =n_{eo} && \quad \text{uniaxial} \\
           &   n_1 \neq n_2  \neq n_3      & & \quad \text{biaxial} 
\end{align}
In this chapter, we will only discuss  uniaxial crystals. Their two index values are called 'ordinary' and 'extraordinary'. The principal axis of the third, extraordinary value is called 'optics axis', not to be confused with the optical\sidenote{In German, both is 'optische Achse'!} axis ($=z$).



\section{Propagation along a principal axis}

Let us start discussing the propagation of light in anisotropic media by assuming that we travel along a principal axis. This means that the wavevector $\bk$ is parallel to one of the $x_i$. When the electric field $\bE$ is parallel to another principal $x_j$, then also $\bD$ remains parallel to $x_j$. The wave travels as in an homogenous medium with index of refraction $n_j$ and the state of polarization remains unchanged.

When the direction of the electric field does not coincide with one of the $x_i$, then we can write it as superposition of fields along several $x_i$. Each of these fields travels as above, but  experiences a different index of refraction. After a distance $d$ they  have acquired a phase difference of 
\begin{equation}
    \Delta \phi = \Delta n \, k_0 \, d
\end{equation}
This is what a wave plate is doing. A uniaxial crystal is cut such that the in-plane directions have a different index of refraction.


\section{Propagation along an arbitrary direction}

Now the plane wave is allowed to travel in any direction $\bk$. It is not constrained to a principal axis as above. We draw\sidenote{see \cite{SalehTeich1991} for a proof that this works} the wave vector $\bk$ in the index ellipsoid and construct a plane perpendicular to it that contains the origin of the coordinate system. The ellipsoid crosses this plane, forming an \emph{index ellipse}. The principal axes of this index ellipse are perpendicular to $\bk$ and describe the two linear polarization directions that propagate unperturbed, i.e. are normal modes. The lengths of the semi-axes of the ellipse give the index of refraction experienced by these two polarization states.


\begin{marginfigure}
    \inputtikz{\currfiledir ellipsoid_cut}
    \caption{When traveling in direction $\boldsymbol{\hat{u}}$, the refractive index of the eigen-modes are found as semi-axes of the ellipse perpendicular to $\boldsymbol{\hat{u}}$ .}
\end{marginfigure}


In the special case of an uniaxial crystal, one wave, the ordinary wave, will always experience the ordinary index of refraction $n_o$. The other, extraordinary  one will experience an index that is a mixture of ordinary and extraordinary. The mixing ratio depends on the angle $\theta$ that the wave vector makes with the optics axis:
\begin{equation}
    \frac{1}{n(\theta)} = \frac{\cos^2 \theta}{n_o^2} + \frac{\sin^2 \theta}{n_{eo}^2}
\end{equation}



\section{Energy flow and wave fronts}

The extraordinary wave is really extraordinary. We will see this when looking at the direction of the Poynting vector. To describe an optical plane wave, we have the wave vector $\bk$, the fields $\bE$, $\bD$, $\bB$, and $\bH$ and the Poynting vector $\bS$. When the spatial variation of all fields is proportional to $\exp(i \bk \cdot \br)$, then Maxwells equations lead to
\begin{align}
    \bk \times \bH = & - \omega \bD  \label{eq:6_kHD}\\
    \bk \times \bE = &  \omega \mu_0 \bH     \label{eq:6_kEH}
\end{align}
This means that $\bk$, $\bH$, and  $\bD$ are perpendicular to each other, and  $\bk$, $\bE$, and $\bH$. The Poynting vector $\bS = \frac{1}{2} \bE \times \bH^\star$ is perpendicular on $\bE$ and $\bH$.

\begin{marginfigure}
    \inputtikz{\currfiledir energy_flow}
    \caption{For an extraordinary wave, Poynting vector $\bS$ and wave vector $\bk$ point in different directions.
    \label{fig:5_energy_flow}}
\end{marginfigure}

For the extraordinary wave, the tensorial nature of $\beps$ does  comes into play: $\bE$ and $\bD$ are not necessarily parallel. Luckily, this is also not  requited by any relation in the last paragraph. Fig.~\ref{fig:5_energy_flow} sketches how the required perpendicularities are resolved with non-parallel  $\bE$ and $\bD$. This means that the wave vector $\bk$ is not parallel anymore to the Poynting vector $\bS$. As the wave fronts are perpendicular to $\bk$, in the case of extraordinary waves, the energy transport ist not perpendicular to the wavefronts anymore! This justifies the label 'extraordinary'. In the following, I will call this object of wavefronts not perpendicular to energy flow a 'beam', in contrast to a plane wave, and similar to a Gaussian beam, in which also the wave fronts are curved and thus not everywhere perpendicular to the energy flow.

Lets look at the direction of energy flow and wave fronts a bit more in detail. We combine the  equations  \ref{eq:6_kHD} and  \ref{eq:6_kEH} with \ref{eq:5_eps_trensor} and get
\begin{equation}
    \bk \times ( \bk \times \bE) + \omega^2 \mu_0 \epsilon_0 \beps \bE = 0
\end{equation}
At a given frequency $\omega$, this is a linear equation system for the thee components of $\bE$. It has a solution if the determinant vanishes. In the case of uniaxial crystals, this reads
\begin{equation}
    (k^2 - n_0^2 k_0^2) \left( \frac{k_1^2 + k_2^2}{n_{eo}^2} + \frac{k_3^2}{n_o^2} - k_0^2  \right) = 0
\end{equation}
This equation describes a surface in the 3d space of $\bk$, the k-surface. In fact, we find two solutions: the first factor becomes zero, leading to a spherical surface related to  the ordinary wave 
\begin{equation}
    | \bk| = k = n_o \, k_0
\end{equation}
The second factor describes an ellipsoid and the extraordinary wave
\begin{equation}
    \frac{k_1^2 + k_2^2}{n_{eo}^2} + \frac{k_3^2}{n_o^2} = k_0^2 
\end{equation}



How does one use these k-surfaces? We start by defining the direction $\boldsymbol{\hat{u}}$ of the wavevector $\bk$. The distance of the surface from the origin in the direction of $\boldsymbol{\hat{u}}$ gives the index of refraction of that wave. This recover its dependence on angle $\theta$ between $\bk$ and the optics axis $x_3$, as discussed in the last section. Additionally, one can show\footcite{SalehTeich1991} that the Poynting vector $\bS$ is perpendicular to the surface at these points. And the wave fronts are perpendicular to $\bk$ and thus $\boldsymbol{\hat{u}}$ as usual.

XXX fig sketches


\section{Birefringence}

The existence of both an ordinary and an extraordinary wave leads to birefringence. A plane wave is diffracted into two different directions when entering an anisotropic material from the air side. Depending on the polarization direction, the wave sees the ordinary index of refraction $n_o$, or the extraordinary $n_{eo}$, where the latter depends on the  angle relative to the optics axis of the crystal. For both polarization directions, Snell's law has to hold, i.e.
\begin{equation}
    \sin \theta_{air} = n_o \sin \theta_o \quad \text{and} \quad   \sin \theta_{air} = n_{eo} \sin \theta_{eo} 
\end{equation}
When $\theta_a$ is the angle of the optics axis with the surface normal, then we can calculate $n_{eo} = n(\theta_a + \theta_{eo})$, using eq. XXX above.

Snell's law describes the direction of the wave vector $\bk$ and the wave fronts, but it does not describe the energy flow or the direction of the Poynting vector $\bS$, at least not in the case of anisotropic media. We see this when letting the plane wave impinge perpendicular on the crystal surface, i.e., set $\theta_{air} = 0$. So also inside the crystal, the wave fronts remain parallel to the crystal surface, all angles remain equal to zero. But the Poynting vector is perpendicular to the k-surface. When the optics axis of the crystal is neither parallel nor perpendicular to the crystal surface, the Poynting vector for the extraordinary beam will deviate. A beam with the extraordinary polarization direction will be displaced in an anisotropic plate. This is used in different types of polarizing beam-splitter prisms.


XXX Fig. 6.32 S/T


XXX discuss pol BS ???

\section{Optical activity and magneto-optics}

I would like to briefly discuss another aspect of anisotropic media. Until now we have assumes that the dielectric tensor $\beps$ is symmetric, i.e. $\epsilon_{ij} = \epsilon_{ji}$. This has led to linear polarization states as eigen-modes, i.e. linearly polarized light travels unperturbed for some polarization directions and directions of travels. Now we do something else. We assume that the dielectric tensor $\beps$ is hermitian, i.e. $\epsilon_{ij} = \epsilon_{ji}^\star$. This leads to circular polarization as eigen-modes. RCP and LCP light travels unperturbed, but with a different index of refraction.

We can write linear polarized light (angie $\theta$) as  superposition of RCP and LCP:
\begin{equation}
    \begin{pmatrix}
        \cos \theta \\ \sin \theta
    \end{pmatrix}
    = 
    \frac{1}{2} e^{-i \theta} 
    \begin{pmatrix}
        1 \\ +i 
    \end{pmatrix}
    +
    \frac{1}{2} e^{+i \theta} 
    \begin{pmatrix}
        1 \\ -i 
    \end{pmatrix}
\end{equation}
After traveling a distance $d$, RCP and LCP waves have accumulated a different phase factor, $\phi_+$ and $\phi_-$
\begin{equation}
    \phi_\pm = 2 \pi n_\pm \frac{d}{\lambda}
\end{equation}
The difference $\phi =\phi_- - \phi_+$ leads to a rotation of the linear polarization state. The resulting Jones vector is
\begin{equation}
    e^{-i (\phi_+ + \phi_-) / 2} 
    \begin{pmatrix}
        \cos \theta  - \phi/2\\ \sin \theta - \phi/2
    \end{pmatrix}
\end{equation}
The 'rotary power' is, for small $\zeta \ll n = \sqrt{\epsilon_{ii}}$
\begin{equation}
    \rho = \frac{\phi/2}{d} \approx - \frac{\pi \zeta}{n \lambda_0}
\end{equation}

Such a tensor is connected with two effects: Optical activity and magneto-optics. Optical active media can be described by 
\begin{equation}
    \bD = \epsilon_0 \epsilon \bE + i \epsilon_0 \boldsymbol{G} \times \bE \quad \text{with} \quad \boldsymbol{G} = \zeta \bk
\end{equation}
with the gyration vector $\boldsymbol{G}$. When the wave travels in z-direction, then only the off-diagonal elements $\epsilon_{12} = i \zeta = \epsilon_{21}^\star$ are different from zero and the diagonal tensor elements are all the same. This can be found in materials with a helical structure, for example quartz, or liquids of chiral molecules, for example amino-acids or sugars.


The Faraday effect is a similar effect in magneto-optics. The dielectric tensor has the same structure, except that the gyration vector is
\begin{equation}
    \boldsymbol{G} = \gamma \bB
\end{equation}
with the gyromagnetic ratio $\gamma$. The fundamental difference is that an optically active medium rotates the direction of polarization as a function of the direction of propagation, while the Faraday effect is a function of the direction of the magnetic field. Consequently, when a wave is reflected and travels back through the same medium, the effect of optical activity is canceled, while the Faraday effect is doubled. Normally, in optics, the direction of travel can be reversed and everything happens in the opposite direction, as we have done when tracing light rays through lenses. When the Faraday effect is involved, this does not work. Running back does not give the starting condition anymore. In this way, one can build an optical diode, the \emph{Faraday isolator}, which allows light to pass only in one direction and prevents, for example, light from being reflected back into a laser cavity.

\section{Hands on: wave plate}

 XXX missing




%--------------------
\printbibliography[segment=\therefsegment,heading=subbibliography]


\part{Interference and Coherence}
\renewcommand{\lastmod}{December 18, 2023}
\renewcommand{\chapterauthors}{Markus Lippitz}

\chapter{Interference}

\section{Overview}


Interference in general

Interferometer: Michelson, Mach-Zehnder, Fabry-Perot, Sagnac

IOR-Bestimmung
FTIR Spewktroskopie
optischer Kreisel


layered media


DBR / dieelctric filter : examples





\section{Optics of layered media: Transmission and scattering matrix}
We present here the T-matrix method, a versatile technique for studying the optical properties of layered media, i.e. stacks of unstructured films of materials with different dielectric functions. These can be dielectric materials leading to e.g. Bragg reflections and dielectric filters, or metal films leading to surface plasmons. We will discuss transmission and reflection of these stacks.

In a layered medium, a wave traveling through the stack of layers is partially reflected and partially transmitted at each interface. The multiple reflections interfere with each other. To keep track of this, we use in each layer a combined wave traveling in the $+z$ direction and one traveling in the $-z$ direction. These waves mix at interfaces. This formalism is described in chapter 7 of \cite{SalehTeich1991} and \cite{Yeh2005}. A similar formalism with an $E$ and $B$ field traveling in the same direction is described in \cite{Pedrotti2008} and \cite{Macleod2001}.

\begin{marginfigure}
\includegraphics[width=50mm]{\currfiledir T-matrix.png}

\caption{The operation of the transmission matrix
\label{fig:6_T_matrix}}
\end{marginfigure}

Let us assume that we have left of the interface a wave traveling to the right ($+z$ direction) of amplitude $U_1^+$, and one wave traveling to the left of amplitude $U_1^-$. On the right side of the interface, we get the amplitudes $U_2^\pm$ by multiplication with a \emph{transmission} or \emph{transfer} matrix $\mathbf{M}$
\begin{equation}
\begin{pmatrix}
U_2^+ \\ U_2^-
\end{pmatrix}
= 
\begin{pmatrix}
A & B \\ C & D \\
\end{pmatrix}
\cdot
\begin{pmatrix}
U_1^+ \\ U_1^-
\end{pmatrix}
%
= \mathbf{M}
\begin{pmatrix}
U_1^+ \\ U_1^-
\end{pmatrix} \quad . \label{eq:6_def_T_matrix}
\end{equation}
Below we will derive transmission matrices $\mathbf{M}_i$ for every interface and the homogeneous space in between. The full stack can then be described by a product matrix, multiplying together all partial matrices $\mathbf{M}_i$ along the stack
\begin{equation}
\mathbf{M}_\text{total} = \mathbf{M}_n \cdot  \mathbf{M}_{n-1} \cdots\mathbf{M}_2 \cdot  \mathbf{M}_{1} \quad . 
\end{equation}
This is a very convenient feature of the transmission matrix.
Note that we label the interactions from left to right with $1$ to $n$, but the matrices are multiplied from right to left, as mathematics has it origin in Arabic culture.



An inconvenient feature of the transmission matrix is that its matrix element have no direct physical meaning. The problem is that we multiply on the matrix a vector that is half an input, half an output of this interface. We know what comes out (travels to the left), and the matrix should tell us what comes in from the other side. In this sense, the related \emph{scattering} matrix $\mathbf{S}$ is closer to physical meaning:
\begin{equation}
\begin{pmatrix}
U_2^+ \\ U_1^-
\end{pmatrix}
= 
\begin{pmatrix}
t_{12} & r_{21}  \\ r_{12} & t_{21}
\end{pmatrix}
\cdot
\begin{pmatrix}
U_1^+ \\ U_2^-
\end{pmatrix}
%
= \mathbf{S}
\begin{pmatrix}
U_1^+ \\ U_2^-
\end{pmatrix} \quad . 
\end{equation}
The scattering matrix connects waves traveling towards the interface with those traveling away from the interface. The entries $t_{ij}$ and $r_{ij}$ are the transmission and reflection coefficients for the amplitudes of the waves traveling from $i$ to $j$ (i.e. $12$ is traveling towards the right, $+z$ direction). However, for the scattering matrix $\mathbf{S}$, the full stack can not be calculated by multiplying together all partial matrices.

\begin{marginfigure}
\includegraphics[width=50mm]{\currfiledir S-matrix.png}

\caption{The operation of the scattering matrix
\label{fig:6_S_matrix}}
\end{marginfigure}

It is therefore convenient to switch between both representations, derive the scattering matrix $\mathbf{S}$ for each situation, and then convert into a transmission matrix $\mathbf{M}$. The relations are\sidenote{\cite{SalehTeich1991}  eq. 7.7}
\begin{align}
\mathbf{M} =  &
\begin{pmatrix}
A & B \\ C & D \\
\end{pmatrix}
=
\frac{1}{t_{21}}
\begin{pmatrix}
t_{12} t_{21} - r_{12}r_{21} & r_{21} \\ - r_{12} & 1 \\
\end{pmatrix} \label{eq:6_M_from_S}
\\
\mathbf{S} =  &
\begin{pmatrix}
t_{12} & r_{21}  \\ r_{12} & t_{21}
\end{pmatrix}
=
\frac{1}{D}
\begin{pmatrix}
AD - BC & B \\ -C & 1 \\
\end{pmatrix} \label{eq:6_S_from_M}
\end{align}
as long as $D$ or $t_{21}$ are not zero.


The transmission in backward direction $t_{21}$ is thus the reciprocal of the $D$-element of $\mathbf{M}_\text{total} $. The transmission in forward direction is 
\begin{equation}
t_{12} = \frac{\text{det } \mathbf{M}_\text{total} }{D}
\end{equation}
and similar for the reflection from the front side
\begin{equation}
r_{12} = - \frac{C }{D} \quad . 
\end{equation}




\section{Electrical fields}

We need to define the physical meaning of the amplitudes $U_i^\pm$ to be able to calculate the reflection ($r_{ij}$) and transmission ($t_{ij}$) coefficients. We assume plane waves 
\begin{equation}
\mathbf{E} \, e^{i (\mathbf{k}  \cdot \mathbf{r} - \omega t)}
=
\mathbf{\hat{E}} \, U \, e^{i \, k_z z} \, e^{i \, k_x x} \, e^{-i \omega t}
\end{equation}
where the wave vector $\mathbf{k} $ lies in the $xz$-plane, $U$ defines the amplitude of the wave and $\mathbf{\hat{E}} $  the polarization direction.
With   the full length of the wave vector in vacuum $k_0 = 2 \pi / \lambda$ and the refractive index $n$ of the medium we get
\begin{equation}
k_{z}^2 + k_{x}^2  = n^2 \, k_0^2  \quad . 
\end{equation}
The polarization directions are
\begin{equation}
\mathbf{\hat{E}}^{(s)} = \begin{pmatrix}
 0 \\ 1 \\ 0 \\
\end{pmatrix}
\quad 
\text{and}
\quad
\mathbf{\hat{E}}^{(p)} =\frac{1}{n \, k_0} \begin{pmatrix}
\pm k_z \\ 0 \\  k_x  \\
\end{pmatrix} \quad . \label{eq:6_Esp_def}
\end{equation}
The $\pm$-sign takes the sign of the direction of travel, see Fig. 2.2 in \cite{Novotny-Hecht2012}. Note that with this definition we have $|\mathbf{\hat{E}}| = 1$, which differs from problem 12.4 in \cite{Novotny-Hecht2012}.

The left and right traveling waves are thus
\begin{equation}
\mathbf{E}^+ = \mathbf{\hat{E}} \, U^+ \, \, e^{+ i \, k_z z}
\quad
\text{and}
\quad
\mathbf{E}^- = \mathbf{\hat{E}} \, U^- \, \, e^{- i \, k_z z}
\end{equation}
where we have split off the global term $ e^{i \, ( k_x x - \omega t)}$.

\section{Propagation matrix}

Before we come to interfaces, let us discuss the transmission matrix of a homogeneous material layer $j$ of thickness $d_j$ and (complex) refractive index $n_j$. Relevant is the $z$-component of the (complex) wave vector $k_{z,j}$. Note that we do \emph{not} use the sign of  $k_{z,j}$ to describe the direction of travel.
Independent of the propagation direction, each wave sees a reflection coefficient $r=0$ and a (complex) transmission coefficient $t$
\begin{equation}
t = t_{12} = t_{21} = e^{+ i \, k_{z,j} \, d_j } \quad . 
\end{equation}
The transmission matrix of a homogeneous medium is thus
\begin{equation}
\mathbf{M} = 
\begin{pmatrix}
e^{+i \, k_{z,j} \, d_j } & 0 \\0 & e^{-i \, k_{z,j} \, d_j } \\
\end{pmatrix} \quad . 
\label{eq:6_M_prob}
\end{equation}


\section{Interface matrix}

The transmission and reflection coefficients of an interface are the Fresnel coefficients $r$ and $t$ for s and p polarization, as defined in chapter \ref{chap:dielectrics}. We assume non-magnetic materials ($\mu = 1$).


% We follow here \cite{Novotny-Hecht2012}, who follow \cite{BornWolf2002}, especially in the direction of the field vectors, see Fig. 2.2 in \cite{Novotny-Hecht2012}. In this definition,  $r^s$ and $r^p$ differ at normal incidence by a factor of $-1$. We assume non-magnetic materials ($\mu = 1$) and get for a wave traveling from medium 1 towards medium 2
% \begin{align}
%  r_{12}^s = & \frac{k_{z,1} - k_{z,2}}{k_{z,1} + k_{z,2}}  = - r_{21}^s\\
%  t_{12}^s = & \frac{2 \, k_{z,1}}{k_{z,1} + k_{z,2}} =  \frac{k_{z,1}}{k_{z,2}}  \,  t_{21}^s\\
%   r_{12}^p = & \frac{\epsilon_2	 k_{z,1} - \epsilon_1 k_{z,2}}
% 				  {\epsilon_2 k_{z,1} + \epsilon_1 k_{z,2}}  = - r_{21}^p\\
%   t_{12}^p = & \frac{2 \sqrt{\epsilon_1 \epsilon_2}	 \,k_{z,1} }
% 				  {\epsilon_2 k_{z,1} + \epsilon_1 k_{z,2}}  = \frac{k_{z,1}}{k_{z,2}}  \,  t_{21}^p \quad . 
% \end{align}
% We could also write these coefficients in terms of angle of incidence $\theta$ with
% \begin{equation}
%  \theta = \arcsin \frac{k_x}{n k_0} = \arcsin \sqrt{1 - \left( \frac{k_z}{n k_0} \right)^2 } \quad . 
% \end{equation}
% This would also hold in the case of evanescent waves ($k_x > n k_0$) when we allow complex angles $\theta$. We nowhere need that $\theta$ is a geometrical angle. We only need that $n \sin \theta$ is the same for all layers.

With eq.~\ref{eq:6_M_from_S} we get for both polarization directions the transmission matrix
\begin{equation}
\mathbf{M}_{12} = \frac{1}{t_{21}} 
\begin{pmatrix}
1 & r_{21} \\ r_{21} & 1 \\
\end{pmatrix} \quad ,
\end{equation}
as 
\begin{equation}
t_{12} t_{21} - r_{12}r_{21} = t_{21}^2 \frac{k_{z,1}}{k_{z,2}} + r_{21}^2 = 1 \quad . 
\end{equation}
Note that the transmission matrix from medium 1 to medium 2 uses the Fresnel coefficients of the backwards direction!
We can abbreviate this to\sidenote{In problem 12.4 in \cite{Novotny-Hecht2012} the leading $1/\eta$ seems to be missing!} (see also appendix at the end of this chapter)
\begin{equation}
\mathbf{M}_{12} 
=\frac{ 1}{2 \eta }
\begin{pmatrix}
1 + \kappa & 1  -\kappa \\  1  - \kappa  & 1 + \kappa \\
\end{pmatrix} \label{eq:6_M_kappa}
\end{equation}
with 
\begin{equation}
\kappa = \eta^2 \,
\frac{  k_{z,1} }{ k_{z,2}}
\quad
\text{and}
\quad
\eta^s = 1 \quad \text{or} \quad \eta^p = \sqrt{ \frac{\epsilon_2}{\epsilon_1} } \quad . 
\end{equation}
The factors $\eta$ in front of the transmission matrix $\mathbf{M}_{12} $ can be collected in front of the total transmission matrix $\mathbf{M}_\text{total}$, in case one is not interested in the distribution of the fields inside the stack. Then, all $\eta^p$ collapse into $\sqrt{\epsilon_\text{first} / \epsilon_\text{last}}$, which is equal to one in case the terminating half-spaces of the layered medium have both the same dielectric constant. 



\section{Appendix: derivation of eq. \ref{eq:6_M_kappa}}


We start from 
\begin{equation}
\mathbf{M}_{12} = \frac{1}{t_{21}} 
\begin{pmatrix}
1 & r_{21} \\ r_{21} & 1 \\
\end{pmatrix}
\end{equation}
and abbreviate the Fresnel coefficients as
\begin{align}
  r_{21}^s = & \frac{k_{z,2} - k_{z,1}}{k_{z,1} + k_{z,2}}  = \frac{b - a}{a + b} \\
 t_{21}^s = & \frac{2 \, k_{z,2}}{k_{z,1} + k_{z,2}} =   \frac{2 b \eta }{a + b}   \\
  r_{21}^p = & \frac{\epsilon_1	 k_{z,2} - \epsilon_2 k_{z,1}}
				  {\epsilon_2 k_{z,1} + \epsilon_1 k_{z,2}}  =   \frac{b - a}{a + b}\\
  t_{21}^p = & \frac{2 \sqrt{\epsilon_1 \epsilon_2}	 \,k_{z,2} }
				  {\epsilon_2 k_{z,1} + \epsilon_1 k_{z,2}}  =   \frac{2 b  \eta }{a + b} 
\end{align}
with $a = \epsilon_2 k_{z,1}$, $b =     \epsilon_1 k_{z,2}$ and $\eta = \sqrt{\epsilon_2 / \epsilon_1}$. In the case of s-polarization, the $\epsilon_i$ are ignored / set to one. With this we get
\begin{align}
\mathbf{M}_{12} = & \frac{a+b}{2 b \eta} 
\begin{pmatrix}
1 & (b-a)/(a+b) \\  (b-a)/(a+b) & 1 \\
\end{pmatrix}
= 
 \frac{1}{2 b \eta} 
\begin{pmatrix}
b+a & b-a \\  b-a & b+a \\
\end{pmatrix} \\
= &
 \frac{1}{2  \eta} 
\begin{pmatrix}
1+\frac{a}{b} & 1- \frac{a}{b} \\  1- \frac{a}{b} & 1+\frac{a}{b} \\
\end{pmatrix}
= 
 \frac{1}{2  \eta} 
\begin{pmatrix}
1+\kappa & 1- \kappa \\  1- \kappa & 1+\kappa \\
\end{pmatrix} 
\end{align}
with 
\begin{equation}
\kappa = \frac{a}{b} = \eta^2 \,
\frac{  k_{z,1} }{ k_{z,2}}
\quad
\text{and}
\quad
\eta^s = 1 \quad \text{or} \quad \eta^p = \sqrt{ \frac{\epsilon_2}{\epsilon_1} } \quad .
\end{equation}


%--------------------
\printbibliography[segment=\therefsegment,heading=subbibliography]

\renewcommand{\lastmod}{January 12, 2023}
\renewcommand{\chapterauthors}{Markus Lippitz}

\chapter{Coherence}


\goal{By the end of this chapter you should be able identify requirements for intereferce contrast.}


\section{Overview}

So far we have always assumed perfectly deterministic light. We wrote something like
\begin{equation}
    u(\br, t) = U(\br) \, e^{-i \omega t} \quad \text{with} \quad U(\br) = \frac{A}{r} \, e^{i \bk \br}
\end{equation}
and could thus determine the field amplitude at each point in space $\br$ for all times $t$. This is what we call 'coherent'. It is in contrast to a light field which contains some randomness. In this chapter we will discuss ways to describe the statistical nature of a partially random light field. As always when randomness is involved, we will not be able to predict every realization of a random process, but only the average and some other statistical properties of an ensemble of realizations. We will define a measure of the average intensity and the 'randomness' of the light field. The latter is called the coherence function or autocorrelation.

We will discuss a relationship between the coherence function and the spectrum of light (the Wiener-Khinchin theorem), and a method for measuring the diameter of a distant star by analyzing the coherence of the detected light (the Michelson stellar interferometer).

\begin{marginfigure}
    \inputtikz{\currfiledir temp_noise}

    \inputtikz{\currfiledir space_noise}

    \caption{Partially coherent waves in space and time.}
\end{marginfigure}
\section{Intensity}

We have written the intensity of a scalar wave (and similar for a vectorial electromagnetic wave) as
\begin{equation}
    I(\br, t) = \left| u(\br, t ) \right|^2 \quad . \label{eq:7_coherent_wave}
\end{equation}
This assumes perfect coherence. If we allow randomness, we have to average over many realizations of the same random process, similar to rolling the dice very often and averaging over the result. We write this as
\begin{equation}
    I(\br, t) =  \left< \left| u(\br, t ) \right|^2 \right>
\end{equation}
where the pointed brackets denote the ensemble average. From now on, we will call $I$ the (average) intensity and $|u|^2$ the instantaneous intensity.

The intensity $I$ can be time-dependent or time-independent. The latter case is a statistically stationary process, which means that the average is independent of time, although each individual realization may still vary with time. The light from a light bulb is an example. In the case of a \emph{stationary process}, we can  write
\begin{equation}
    I(\br) =  \lim_{T \rightarrow \infty} \, \frac{1}{2T} \, \int_{-T}^T \left| u(\br, t ) \right|^2  dt \quad .
\end{equation}



\begin{marginfigure}
    \inputtikz{\currfiledir stationary}    
    \caption{Stationary (top) and non-stationary (bottom) wave. \newline Plotted is $|u(t)]^2$.}
\end{marginfigure}

\section{Temporal coherence}

Let us drop the spatial dependence on $\br$ for a moment and just look at a single point in space. We want to describe the 'randomness' of the complex-valued field $u$, assuming a stationary process. The question is how similar are the amplitudes $u(t)$ now and a time $\tau$ later, i.e. $u(t + \tau)$. The more similar they are, the more memory the process has, the less random it is. We quantify this by the \emph{temporal coherence function} or \emph{autocorrelation function} $G(\tau)$:
\begin{equation}
    G(\tau) = \left< u^\star(t) \, u(t + \tau) \right> = 
    \lim_{T \rightarrow \infty} \, \frac{1}{2T} \, \int_{-T}^T u^\star(t) \, u(t + \tau)  \,  dt
\end{equation}
with the properties
\begin{equation}
    G(\tau) = G^\star(- \tau) \quad \text{and} \quad I = G(0) \quad .
\end{equation}

As an example, consider a field $u$ that is either $+1$ or $-1$. On average, it should be as often positive as negative, i.e. its average $\braket{u(t)} = 0$. The expression $u^\star(t) \, u(t + \tau)$ is positive if $u$ does not change its sign, otherwise it is negative. If the relation between $u(t)$ and $u(t+\tau)$ is completely random, the mean of $u^\star(t) \, u(t + \tau)$ will be zero. On the other hand, if $u$ does not flip within the time $\tau$, $G$ is positive. If $u$ preferentially flips the sign within the time delay $\tau$, $G(\tau)$ will be negative.

It is convenient to remove the intensity from the definition of the coherence function. We define the \emph{degree of temporal coherence} $g(\tau)$ as
\begin{equation}
    g(\tau) = \frac{G(\tau)}{G(0)}
    = \frac{\left< u^\star(t) \, u(t + \tau) \right>}{\left< u^\star(t) \, u(t) \right>}
\end{equation}
so that $g(0) = 1$ and $|g(\tau)| \le 1$. For a coherent wave such as eq.  \ref{eq:7_coherent_wave}, we have  $|g(\tau)| = 1$.

For partially coherent light, $|g(\tau)|$ decreases with increasing delay $\tau$. The \emph{coherence time} $\tau_c$ is a characteristic time of this decrease, describing the width of $|g(\tau)|$. One defines 
\begin{equation}
    \tau_c = \int_{-\infty}^{\infty} |g(\tau)|^2 \, d\tau \quad .
\end{equation}
Together with the speed of light we get a coherence length $l_c = c_0 \, \tau_c$.




\section{Power spectral density}

Lets discuss Fourier transformations of random fields $u(t)$. We could just obtain the Fourier transform $v(\nu)$ of $u(t)$ by
\begin{equation}
    v(\nu) = \int_{-\infty}^\infty u(t) \, e^{i 2 \pi \nu t} \, dt \quad . \label{eq:7_FT_full}
\end{equation}
Since $I = \braket{|u|^2}$ is the intensity (energy per time and area), $\braket{|v(\nu)|^2}$ is an energy spectral density (energy per frequency interval and area). If the process is really stationary, it lasts forever, so its total energy is infinite. It thus makes more sense to truncate the Fourier integral
\begin{equation}
    v_T(\nu) = \int_{-T/2}^{T/2} u(t) \, e^{i 2 \pi \nu t} \, dt
\end{equation}
and defining the \emph{power spectral density} or spectrum  $S(\nu)$ (energy per frequency interval, area and time) as
\begin{equation}
    S(\nu) = \lim_{T \rightarrow \infty} \, \frac{1}{2T} \, 
    \braket{|v_T(\nu)|^2} \quad .
\end{equation}

The interesting point for us in this chapter is that the  spectrum  $S(\nu)$  is the Fourier transform of the autocorrelation function $G(\tau)$. 
\begin{equation}
    S(\nu) = \int_{-\infty}^\infty  G(\tau) \,  \, e^{i 2 \pi \nu \tau} \, d\tau \quad . \label{eq:7_S_G}
\end{equation}
This relation is the \emph{Wiener-Khinchin theorem}. It is rather trivial if the Fourier transform \ref{eq:7_FT_full} exists, i.e. the integral eq. \ref{eq:7_FT_full} converges, because the field $u(t)$ is some kind of non-stationary pulse. The important point of the theorem is that it also works for stationary processes.

As spectrum and autocorrelation function are related by a Fourier transform, also the spectral width $\Delta \nu$ and coherence time $\tau_c$ are related by
\begin{equation}
    \Delta \nu = \frac{1}{\tau_c}
\end{equation}
using a suitable definition of $\Delta \nu$ (see \cite{SalehTeich1991}, eq.11.1-18). The narrower the spectrum of  a light source is, the longer is its coherence length.



\section{Interference and temporal coherence}

Let us consider a partially coherent wave $u(t)$ described by its autocorrelation function $g(\tau)$, traveling through a Michelson interferometer. At the symmetric output of the interferometer, the total field is $u(t) + u(t + \tau)$, dropping the reflection and transmission coefficients  $r t$ in the prefactor. The arm length difference $d$ defines the time lag $\tau = 2 d / c_0$. We measure the \emph{interferogram}, the relation between $I$ and $\tau$
\begin{align}
    I(\tau) = & \braket{|u(t) + u(t + \tau)|^2} \\
    = &  2 \braket{|u|^2} + \braket{u^\star(t) u(t + \tau)} + \braket{u(t) u^\star(t + \tau)} \\
    = & 2 I_0 + 2 \Re\{G(\tau)\} \\
    = & 2 I_0 \left( 1 + |g(\tau)| \cos [ \arg \{ g(\tau )\} ] \right)  \label{eq:7_intfer_g}
\end{align}
where $\arg\{|a| e^{i \phi} \} = \phi$. For a perfectly coherent wave with frequency $\omega$, we have $g(\tau) = e^{-i \omega \tau}$ and recover the usual result. For a partially coherent wave, $|g(\tau)|$ drops in amplitude with delay $\tau$, so that the oscillation fringes reduce in amplitude. We call the \emph{visibility} $\mathcal{V}$ (or modulation contrast or depth)
\begin{equation}
    \mathcal{V} = \frac{I_\text{max}- I_\text{min} }{I_\text{max}+ I_\text{min} } = |g(\tau)| \quad .
\end{equation}
The magnitude of the autocorrelation gives the contrast of the interference fringes; its phase shifts the peaks relative to the perfectly coherent case. 


\begin{marginfigure}
    \inputtikz{\currfiledir interferogram}    
    \caption{Interferogram of a partially coherent wave.}
\end{marginfigure}

\section{Fourier-transform spectroscopy}

Using the Wiener-Khinchin theorem in reverse direction
\begin{equation}
    G(\tau) = I_0 g(\tau) = \int_0^\infty S(\nu) e^{-i 2 \pi \nu \tau} \, d\nu
\end{equation}
and noting that $S(\nu)$ is real, we can write eq.  \ref{eq:7_intfer_g} as
\begin{equation}
    I(\tau) = 2 \int_0^\infty S(\nu) \left[1 + \cos( 2 \pi \nu \tau ) \right] \, d\nu \quad .
\end{equation}
This can be reversed to obtain the power spectral density $S(\nu) $ from the interferogram $I(\tau)$
\begin{equation}
    S(\nu) = 2 \int_0^\infty \left[ I(\tau) - \frac{1}{2} I(0) \right] \cos( 2 \pi \nu \tau ) \, d\tau \quad .
    \label{eq:7_FTIR}
\end{equation}
Measuring the spectrally integrated transmitted power through a Michelson interferometer as a function of delay $\tau$ and roughly Fourier transforming the result gives the spectrum of the light source.

How does this happen? The spectral transmission function of a Michelson interferometer is similar to that of a Fabry-Perot interferometer, a $\cos^2$ function where the distance between the peaks increases with decreasing path length difference $d$. When we transmit our unknown light spectrum $S(\nu)$ through the Michelson interferometer, we multiply it by this transmission function. Measuring the total power is a spectral integration. This is what a cosine transform does. Eq.  \ref{eq:7_FTIR} simply reverses the operation of the interferometer.

Fourier transform spectroscopy is particularly useful in cases where no other spectrometer can be used, such as in the infrared spectral range. This is called Fourier Transform Infrared (FTIR) spectroscopy. In this spectral region, it is technically challenging to build bright light sources and efficient detectors. The advantage of FTIR is that you only need one detector (not an array of detectors) and about half of the light power is always incident on that detector. In a grating spectrometer, on the other hand, the total power is distributed over many pixels. However, controlling the delay $\tau$ accurately enough is a challenge.



\section{Spatial coherence}


Similar to two points $t$ and $t + \tau$ in time we can also compare two points $\br_1$ and $\br_2$ in space. We define spatial autocorrelation functions
\begin{equation}
    G(\br_1, \br_2) = \braket{u^\star(\br_1,t) \,  u(\br_2, t)} \label{eq:7:G_space}
\end{equation}
and
\begin{equation}
    g(\br_1, \br_2) = \frac{  G(\br_1, \br_2)}{\sqrt{I(\br_1) \, I(\br_2)}} \quad .
\end{equation}
Again, the coherence or autocorrelation typically decreases with the distance between $\br_1$ and $\br_2$. The \emph{coherence area} $A_c$ is an area within which $g(\br_1, \br_2)$ has only dropped to a critical value. For a hot emitting surface, the coherence area is about $\lambda^2$, so in most cases this source can be assumed to be incoherent. If the coherence area is larger than, say, the size of a pinhole, the beam can be assumed to be fully spatially coherent.

Sometimes it is useful or necessary to combine spatial and temporal coherence:
\begin{equation}
    g(\br_1, \br_2, \tau ) = \frac{ \braket{u^\star(\br_1,t) \,  u(\br_2, t + \tau)}}{\sqrt{I(\br_1) \, I(\br_2)}} \quad .
\end{equation}



\section{Double-slit experiment with partially coherent waves}

Above, we used a Michelson interferometer to test and demonstrate the effect of temporal coherence by interfering a wave with a time-shifted copy of it. Now we do the same with spatial coherence using a Young double-slit experiment.

A partially coherent wave $u$ is described by its autocorrelation function $g(\br_1, \br_2, \tau )$. It hits an opaque screen with two small spherical holes at positions $\br_{1,2} = (\pm a, 0,0)$. Each hole results in a diffracted spherical wave. On a screen at distance $d$ we find an interference pattern which we observe along a line $\br = (x, 0, d)$. For simplicity, we assume that the intensity $I_0$ of both spherical waves is the same.

At point $x$ along our line of observation $\br$, two waves interfere. We calculate the difference $\tau$ in the travel time between the hole at $\br_{1,2}$ and the screen at $\br$, since this time difference $\tau$ enters the coherence function. By simple geometry we get
\begin{equation}
    \tau = \frac{| \br - \br_1| - | \br - \br_2|}{c_0} = \frac{ (x+a)^2 - (x-a)^2}{2d c_0} = 
    \frac{2ax}{d c_0} = \frac{\theta}{c_0} \, x
\end{equation}
where $\theta$ is the angle between the holes as seen from the screen. Following the same scheme as above with the Michelson interferometer, we get
\begin{equation}
    I(x) = 2 I_0 \left( 1 + |g(\br_1, \br_2, \tau)| \cos [ \arg \{ g(\br_1, \br_2, \tau)\} ] \right) \quad .
\end{equation}
As above, the visibility $\mathcal{V}$ of the fringes drops with deceasing coherence. The characteristic decay length is coherence length divided by the opening angle $\theta$. The period of the spatial oscillation is $\bar{\lambda}/ \theta$.



\begin{marginfigure}
    \inputtikz{\currfiledir young_fringes}    
    \caption{Young double slit interference of a partially coherent wave, assuming very small holes.}
\end{marginfigure}


\section{Gain of spatial coherence by propagation}

A spatially incoherent light source can result in spatially coherent light! One way to achieve this is simply to let it propagate long enough. We can divide the light source into patches of the size of the spatial coherence area, $A_c$. Within this area (which can be just a point) the source is coherent. Since the area has a finite size, the light is diffracted as it propagates, covering a larger and larger area.  At a distance, neighboring points on a screen will see a very similar mixture of fields from the individual coherence areas of the source. Spatial coherence thus has increased.

Lets look at this more formally. We use the impulse response $h$ to connect the field $u_1$ in the source plane with the field $u_2$ in the target plane, as introduced in the chapter \ref{chap:Fourier} on Fourier optics (see, e.g. eq. \ref{eq:3_h_free_space})
\begin{equation}
    u_2(\br) = \int h(\br, \br') \, u_1(\br') \, d\br'  \quad .
\end{equation}
Using the definition of spatial coherence function $G(\br_1, \br_2)$ (eq. \ref{eq:7:G_space}), we can calculate the spatial coherence in the target plane as function of the coherence in the source plane
\begin{equation}
    G_2(\br_1, \br_2) = \iint  h^\star(\br_1, \br'_1) \,  h(\br_2, \br'_2) \,  G_1(\br'_1, \br'_2) \, d\br'_1  d\br'_2  \quad .
\end{equation}
When the light source is fully incoherent, then $G_1$ is different from zero only for $\br'_1 = \br'_2$ and at these points it is identical to the intensity $I$. Things thus simplify to 
\begin{equation}
    G_2(\br_1, \br_2) = \int  h^\star(\br_1, \br') \,  h(\br_2, \br') \,  I(\br') \, d\br'  \quad .
\end{equation}
Now we can use the same approximations that led to the optical Fourier transform by propagation in the Fraunhofer approximation (eq. \ref{eq:3_FT_by_prop}). When propagating in free space over a long distance, each impulse response effectively contributes an exponential function identical to that of a Fourier transform. The normalized coherence function $g_2$ at the target plane is thus the spatial 2D Fourier transform $\mathcal{I}_1$ of the intensity at the source plane.
\begin{equation}
    \left| g_2(x_1, y_1, x_2, y_2) \right| = \frac{\left| \mathcal{I}_1 \left( \frac{x_1 - x_2}{\lambda d}, \frac{y_1-y_2}{\lambda d} \right)  \right|}{\mathcal{I}_1 (0,0)}  \quad .
\end{equation}

\section{Radiation of an incoherent circular source}

As an example, lets look at the sun. We model it as incoherent circular source. Within the circle, it should have a constant intensity. It is  convenient to combine the radius $a$ and the distance $d$ to the apparent opening angle $\theta_s$ of the source as seen from the target plane
\begin{equation}
    \theta_s = \frac{2a}{d}  \quad .
\end{equation}
The 2D Fourier transform of a circle is similar to a sinc-function, replacing the sine by a Bessel function  $J_1$ (see Appendix \ref{chap:appendix_Fourier}). In total we get
\begin{equation}
    \left| g_2(x_1, y_1, x_2, y_2) \right| = \left|  
 \frac{2 J_1( \pi \rho \theta_s / \lambda)}{\pi \rho \theta_s / \lambda}
    \right|
\end{equation}
using $\rho^2 = (x_1-x_2)^2 + (y_1-y_2)^2$. The first zero of the Bessel function defines a characteristic radius in the target plane
\begin{equation}
    \rho_c = 1.22 \frac{\lambda}{\theta_s}  \quad .
\end{equation}
We see our sun under an angle of about $0.5^\circ \approx 8.7 \cdot 10^{-3}$~rad. At a wavelength of 500~nm, the coherence radius $ \rho_c$ is about 70~\textmu m.

To generate spatially coherent light it is thus sufficient to let sunlight shine through about 140~\textmu m diameter pinhole, without any lens or similar. The sun's intensity is about 1~kW/m$^2$ on earth on a bright day, so that about 15~\textmu W pass through the pinhole. To also obtain temporally coherent light, we need to spectrally filter to a narrow enough spectral range (eq. \ref{eq:7_S_G}), which reduces the power further.


\section{Michelson stellar interferometer}

We can also turn the argument around. By measuring the coherence radius $\rho_c$ of a distant star, we can determine its apparent angle $\theta_s$. Together with a known distance $d$ we can thus measure its radius $a$. This is the idea behind the Michelson stellar interferometer. The star Betelgeuse\sidenote{dt. Beteigeuze} in the constellation Orion ($\alpha$-Orionis) is one of the brightest stars in our northern sky.  At a wavelength of 570~nm  a coherence radius of $\rho_c = 3.1$~m was measured, corresponding to a source angle of $\theta_s = 22.6 \cdot 10^{-8}$~rad. The distance to Betelgeuse is 548 light-years ($5.2 \cdot 10^{18}$~m), so its diameter is $1.2 \cdot 10^{12}$~m, which is within 20~\% of the Wikipedia value. Betelgeuse is a red supergiant. Its diameter is 1500 times larger than the Sun, or about 7 times larger than the Earth's orbit around the Sun.

How do you do this in practice? We need to collect light from the star at two points on Earth, separated by a variable distance $\rho$, and determine the visibility $\mathcal{V}$ of the interference fringes when the light beams overlap. The advantage is that one does not need an optical telescope of diameter $\rho$. Even the optical telescopes at distance $\rho$ need not be of exceptionally high quality, since we only need to collect light, not resolve the star. As long as the star is much brighter than its immediate surroundings, even simple searchlights have been used.

The collected light must be directed by mirrors to a central position and interfered. Fluctuations in the atmosphere will lead to a phase difference between the two telescopes, which will randomly shift the position of the fringes. But we are not interested in the position of the fringes, only in their contrast $\mathcal{V}$, so even these fluctuations can be tolerated.\sidenote{see \cite{Brooker_Optics} for a discussion}


\section{Intensity autocorrelation}

To detect interference fringes in the Michelson stellar interferometer, the light from the two remote telescopes must be brought to a common point without losing the phase relationship. This is possible, but inconvenient. Here the intensity (or second order)
autocorrelation comes in handy, as R.~Hanbury~Brown and R.~Q.~Twiss demonstrated in 1956 (\cite{HanburyBrown1956}). We define
\begin{equation}
    G^{(2)}(\tau) = \braket{u^\star(t) \,  u^\star(t+\tau) \,  u( t) u( t + \tau)} 
    =  \braket{I(t) \,  I(t+\tau) } 
\end{equation}
and label all our old (amplitude) correlations as $G^{(1)}$ or $g^{(1)}$. So we are correlating intensities, not fields. You have to think a bit about the term 'intensity' here. It means that we average over some time to get the envelope modulated by $e^{i \omega t}$, but that we do not average too much to preserve the fluctuations. Only the brackets average over a long time.

Since second-order autocorrelation is also based on the same fields as first-order autocorrelation, there is a relationship between the two, at least for classical (chaotic) light\sidenote{light from a laser or an atom does not follow this relationship}.
\begin{equation}
    g^{(2)}(\tau) - 1 = \left|  g^{(1)}(\tau) \right|^2  \quad .
\end{equation}
It is therefore sufficient to measure the intensity of the light at the two telescopes with sufficient time resolution and then calculate $ g^{(2)}(\tau) $ to determine the diameter of the stars. This effectively replaces the light beam between the telescopes with a cable.



%--------------------
\printbibliography[segment=\therefsegment,heading=subbibliography]


\part{Quantum optics}
\renewcommand{\lastmod}{January 26, 2024}
\renewcommand{\chapterauthors}{Markus Lippitz}

\chapter{Quantum Optics}


\goal{By the end of this chapter, you should be able to explain the results of experiments by the properties of photons. }


\section{Overview}

Optics is the study of the interaction of light and matter. We can use different types of models to describe the light and the matter part of it. In all previous chapters we treated light as a scalar or electromagnetic wave. We have also described matter as a classical Lorentz oscillator. Although we have not used it in this lecture, you have seen in other places how to describe matter (especially electrons) with quantum mechanics. In most cases it is sufficient to use quantum mechanics only for the matter part and to keep a classical description for the light part. For example, the external photoelectric effect requires only that the electron be quantized. Light can be described classically.

In this chapter we will go beyond this. We will discuss the so-called 'second quantization'. We now use quantum mechanics to describe the light part by introducing the photon. We will use this formalism to describe photon anti-bunching in the emission of atoms, the Hong-Ou-Mandel experiment of identical photons at a beam splitter, and entanglement between photons.





\section{The photon}

A photon is the quantum of energy in a single optical mode. This idea goes back to Max Planck and the description of black-body radiation. We assume a resonator cavity that forms the optical mode at an angular frequency $\omega$. The mode is characterized by the associated electric field $\bE_\text{mode}(\br, t)$, the wave vector $\bk$ and a polarization state. The central idea is to quantize the energy of each mode:
\begin{equation}
    E_\text{mode} = \left( n + \frac{1}{2} \right) \, \hbar \omega
\end{equation}
where $n$ is the number of photons in this mode. 

We can connect the photon description of electromagnetic fields to the classical description by defining a vacuum field amplitude. Independent of the description, a mode should contain the same energy. In the dark, i.e. in the state $n=0$, quantum mechanics gives an eigen-energy $E_0 = 1/2 \,  \hbar \omega$. This is what we require also from classical electrodynamics\footcite[chap. 7.5]{Fox}
\begin{equation}
E_0 = 
 \int_\text{cavity} \frac{1}{2} 
 \left( \boldsymbol{H} \cdot  \boldsymbol{B} + \boldsymbol{E} \cdot  \boldsymbol{D} \right) \, d\boldsymbol{r} = 
  \int_\text{cavity}  \epsilon_0 \boldsymbol{E}^2 \, d\boldsymbol{r} = \frac{1}{2} \, \hbar \omega
\end{equation}
so that
\begin{equation}
E_{vac} = \sqrt{\frac{\hbar \omega}{2 \epsilon_0 \, V}} \label{eq:8_evac}
\end{equation}
is the amplitude of the field in the dark vacuum, with $V$ being the volume of the cavity.\sidenote{This is the reason we require a cavity. Otherwise the integral would diverge.} 

A photon carries a \emph{momentum} of 
\begin{equation}
    \boldsymbol{p} = \hbar \bk
\end{equation}
leading to, e.g. recoil when emitting or reflecting a photon. Position and momentum are connected by an uncertainty relation, which also motivates diffraction of photons at apertures.

The photon is a \emph{Boson}. All photons within a mode are identical, but  the decomposition of a given electric field into modes is not unique. As usual for Bosons, the spin is an integer multiple of $\hbar$. For a photon
\begin{equation}
    S = \pm \hbar \quad \text{but not } S = 0. 
\end{equation}
The reason for the exclusion of $S=0$ is that the photon has no rest mass. The two spin states correspond to right and left circular polarized light. In these cases, the spin is oriented parallel (RCP) and anti-parallel (LCP) to the wave vector. Linear polarized light corresponds to a superposition of $S=+\hbar$ and $S=-\hbar$. Photons may also have orbital angular momentum. This is the case when the electromagnetic field depends like $\exp(i \ell \phi)$ on the angle $\phi$ in the cylindrical coordinate system, for example in Laguerre-Gaussian modes. Then $\ell$ is the quantum number and $L = \ell  \hbar$ is the orbital angular momentum.


The \emph{position} of a photon is only known when a detectors reports an event. Before that, we only know a probability density to find a photon, which is proportional to the intensity of the optical wave at this position, i.e.
\begin{equation}
    p(\br) \propto I(\br)  \propto | u(\br) |^2 \quad .
\end{equation}
A beam splitter thus does not split a photon. It only splits the probability density to detect it at one output or the other.


\section{Photon stream}

For visible light, 1~W corresponds to about $3 \cdot 10^{18}$ photons per second. So in a beam of 1~nW, we have \emph{on average} 3 photons per nanosecond. Fast photodetectors have a time resolution of about 1~ps. Such a detector would see 0.003~photons per picosecond, i.e. most of the time it would detect nothing and every now and then a photon. The intensity of a beam only determines the average photon rate. The properties of the light source determine the photon statistics, which can be described by the probability of finding $n$ photons in a time interval $T$ or by the intensity correlation function $g^{(2)}(\tau)$ (see end of last chapter). In the following, we discuss first thermal light of low coherence time and coherent laser light, and later the emission of a single atom or dye molecule.

\section{Photon stream of thermal light}

Thermal light of a black body in a cavity is in thermal equilibrium. The probability $p$ to find the energy $E_n$ in the optical mode follows the Boltzmann distribution
\begin{equation}
    p(E_n) \propto \exp \left(  - \frac{E_n}{k_B T} \right)
\end{equation}
at a temperature $T$. With quantized photons, the energy of a mode is determined by the number of photons in it 
\begin{equation}
    E_n = \left( n + \frac{1}{2} \right) \, h \nu
\end{equation}
so that we get for the probability to find $n$ photons in a mode
\begin{equation}
    p(n) \propto  \exp \left(  - \frac{n h \nu}{k_B T} \right) = 
    \left[ \exp \left(  - \frac{h \nu}{k_B T} \right) \right]^n
\end{equation}
where the  $1/2$ hides in the proportionally sign.
We require that the $p(n)$ are normalized, i.e., their sum equals one. We define the mean photon number $\bar{n}$ by
\begin{equation}
    \bar{n} = \frac{1}{\exp( h \nu / k_b T) - 1} \quad .
\end{equation}
Everything together we get a geometric distribution or Bose-Einstein distribution 
\begin{equation}
    p(n) = \frac{1}{\bar{n} + 1} \left(\frac{\bar{n}}{\bar{n} + 1}  \right)^n \quad .
\end{equation}

\begin{marginfigure}
    \inputtikz{\currfiledir thermal}
    \caption{Thermal light. Probability $p(n)$ to find $n$ photons in a time interval $T$ when on average we have $\bar{n} = 0.1$, 1, 5, or 10 photons per time bin.}
\end{marginfigure}

\section{Photon stream of coherent laser light}

A coherent light beam is what we have been assuming all along, until we came to the last chapter's discussion of (in)coherence. It is described by an electric field $\bE(\br, t)$ with angular frequency $\omega$. If $P$ is the power of the beam, the photon flux $\Phi$ (units: photons per second) can be calculated as
\begin{equation}
    \Phi = \frac{P}{\hbar \omega} \quad .
\end{equation}
We now imagine a segment of length $L$ of such a beam. It contains
\begin{equation}
 \bar{n} = \frac{\Phi L}{c} = \frac{P \, l}{\hbar \omega c}
\end{equation}
photons, where we have assumed that $L$ is so large that $\bar{n}$ is well defined, i.e., the granularity of the photons has been averaged out. 

Now we divide the length $L$ into a large number $N$ of segments. We make $N$ so large that the average number of photons per segment is far less than one. In each sub-segment we find either no photons or one photon. The probability of finding a photon is $p = \bar{n} / N$.


Let us reconstruct the whole segment from these sub-segments by taking $N$ sub-segments. What is the probability of ending up with $n$ photons? This is the probability of finding $n$ subsegments with a photon (with probability $p$) and $N-n$ subsegments without a photon (with probability $1-p$). The order does not matter, so we end up with a binomial distribution
\begin{equation}
    P(n) = \frac{N!}{n! (N -n)!} \, p^n (1-p)^{N-n}
 =
 \frac{N!}{n! (N -n)!} \, \left( \frac{\bar{n}}{N}\right)^n \left(1- \frac{\bar{n}}{N} \right)^{N-n}  \quad .
\end{equation}
We let  $N \rightarrow \infty$ and obtain after some math the \emph{Poisson distribution}
\begin{equation}
    P(n) = \frac{\bar{n}^n}{n!} e^{- \bar{n}} \quad \text{for} \quad n = 0, 1, 2, \dots \quad .
\end{equation}
The same distribution also describes the number of events per time interval in a Geiger-Müller counter. The Poisson distribution has the important property that its variance is equal to its mean $\bar{n}$, or its standard deviation is $\sqrt{\bar{n}}$. For $\bar{n} \gtrsim 10$ the Poisson distribution approaches a normal distribution with the same mean and standard deviation. In the logarithmic plot, this appears as inverted parabola. 

\begin{marginfigure}
    \inputtikz{\currfiledir poisson}
    \caption{Coherent light. Same as above. }
\end{marginfigure}


\section{Photon stream of an atom}

The fluorescence emission of a single atom or dye molecule differs in its photon statistics from both thermal light and coherent laser light. This was first observed experimentally by Kimble et al. in 1977 for sodium ions\footcite{Kimble1977} . Let us examine the processes surrounding photon emission in a single atom or molecule described by quantum mechanics. The atom has a ground state and one or more excited states. Optical excitation brings the atom from the ground state to the excited state. This is a statistical process, so it takes some time after the excitation light source is turned on. The excited state decays with some probability to the ground state by emitting a photon. It may also decay without photon emission or into other states. However, after emission of the photon, the atom is certainly in the ground state. And by definition, a ground state cannot decay further, especially it cannot emit another photon. So it takes some time for the excitation light source to bring the atom back to the excited state where a second photon can be emitted. The average time between two emission events is thus given by the excitation rate and the emission rate, and under no circumstances can our quantum mechanical two-level system emit two photons simultaneously. This phenomenon is called \emph{anti-bunching}.

\begin{marginfigure}
    \inputtikz{\currfiledir stream}
    \caption{Sketch of photon detection events over time for thermal (top), coherent (mid), and anti-bunched (bottom) light.}
\end{marginfigure}


\section{Hanburry Brown-Twiss experiment for photons}

We have introduced in the last chapter the intensity autocorrelation function or second-order correlation function that compared intensities $I(t)$ with those shifted in time
\begin{equation}
    G^{(2)}(\tau) = \braket{u^\star(t) \,  u^\star(t+\tau) \,  u( t) u( t + \tau)} 
    =  \braket{I(t) \,  I(t+\tau) }  \quad .
\end{equation}
We can replace the intensity $I(t)$ with the number $n(t)$ of photons detected in the time interval $(t, t+T)$, with the bin integration time $T$
\begin{equation}
    G^{(2)}(\tau)     =  \braket{n(t) \,  n(t+\tau) }  \quad .
\end{equation}
When $T$ approaches the time resolution of the detector, then $n(t)$ is either zero or one. 

To check for anti-bunching, we need to test whether two photons can be detected simultaneously. Typical photodetectors cannot distinguish between one or more photons at the same time. The experimental trick is the Hanburry Brown-Twiss experiment: one splits the photon stream from the atom into two streams and detects each stream with a separate detector. When two photons hit the beam splitter, in some cases they would separate into the two beams. In other cases, they would stay in the same beam, but that does no harm. So we look for coincidences, i.e. both detectors clicking at the same time. This is what $G^{(2)}(0)$ shows. Anti-bunching leads to a dip in the autocorrelation function around $\tau = 0$. The slope of the dip depends on the sum of the excitation and emission rates. 

\begin{marginfigure}
    \inputtikz{\currfiledir HBT}
    \caption{Hanbury Brown--Twiss experiment. The time interval $\tau$ between two photons is determined. \label{fig:8_HBT}}
\end{marginfigure}

We can determine $I(t)$ for light by counting photons within a short interval $T$ and then writing $g^{(2)}(\tau)$ with the counting rate $n(t)$. However, to detect antibunching, $T$ must be very small (about 100 ps). At the same time, averaging requires a long total time, i.e. a large amount of data. 
A more data-efficient approximation is not register the full $n(t)$ trace to calculate $G(\tau)$ (equivalent to all pairs of photons), but to register only the time between  \emph{successive} pairs of photons, which we call $C(\tau)$. 
Finding two photons at distance $\tau$ can happen with a varying number of other photons in between. No photon in between is described by $C(\tau)$. A single photon in between can occur at any time $\tau'$ with $0 < \tau' < \tau$. 
We could integrate over these possibilities. So $C(\tau)$ and $G(\tau)$ are related:
\begin{equation}
    G(\tau) = C(\tau) + \int_0^\tau C(\tau') C(\tau-\tau') d\tau' + \dots \quad .
\end{equation}
Further double, triple, etc. integrals would then describe two, three, etc. photons in between. If the photon rate is small enough or the time of interest $\tau$ is short enough, we can assume $G(\tau) \approx C(\tau)$. So effectively we do not measure $n(t)$ for two detectors in the Hanburry Brown--Twiss experiment, but we start a clock by one detector and stop it via the other, so that we measure $C(\tau)$.

The important point is that anti-bunching only is visible when a single atom or molecule is the source of the photon stream. Averaging over many emitters removes the effect, because sooner or later the emission from one emitter would surely coincide with the emission from another emitter, and the dip would disappear.

\section{Anti-Bunching in semiconductor quantum dots}

An emitter that exhibits antibunching, i.e. emits only a single photon at a time, is called a single-photon emitter or single-photon source. Such light sources are needed for quantum key distribution, as discussed below. Single atoms or dye molecules are single photon emitters. A technologically interesting alternative are quantum dots. These are droplets of a low bandgap semiconductor embedded in a matrix of a higher bandgap semiconductor. In essence, a 3D particle-in-a-box is formed for the electrons and holes. As the size of the droplet approaches the de Broglie wavelength of the electrons, the band structure disappears and the electron states become quantized and discrete, as in an atom or two-level system in general.


For a two-level system, $C(\tau)$ is easy to determine. Immediately after the first photon you are in the ground state with absolute certainty. The excited state is reached with the excitation rate $W_P$, from there back to the ground state with the rate $\Gamma$ of the spontaneous emission. The characteristic time $t_d$ is the reciprocal of the total rate for one cycle.
\begin{equation}
    t_d = \frac{1}{W_P + \Gamma} \quad \text{and so} \quad g^{(2)}(\tau) \approx 1 - a e^{- \tau / t_d} \quad .
\end{equation}
The amplitude $a$ is $a=1$ in the ideal case. In reality, dark noise and background photons cause $a < 1$. However, the case $a> 0.5$ can only be generated by a single photon source or a single two-level system.  

Such an experiment is shown in the figure  \ref{fig:8_gaas_antibunching} for a \ch{GaAs} quantum dot. The optical excitation here was via the surrounding semiconductor, not directly via the exciton. This leads to the 'overshoots' with $g>1$, which are taken into account in the model.

\begin{figure}
    \inputtikz{\currfiledir antibunching}
    \caption{Anti-bunching in a GaAs quantum dot. Data from \cite{Wu2017a}. \label{fig:8_gaas_antibunching}}
\end{figure}




\section{Quantum Key Distribution}

An increasingly important technological application of single-photon sources is the quantum-mechanically secure transmission of an encryption key. There are several ways to encrypt messages. For example, it is possible to exploit the fact that a large number can only be broken down into its prime factors with great effort, but the reverse is easy. However, the complexity of breaking the encryption depends on the available technologies and may be difficult to predict for the future. An encryption that can never be broken is the exclusive-or relation (XOR) of the source text with a key that is the same length as the message and is never used again. This key is called a one-time pad. The recipient needs the same key, performs another XOR with the encrypted message, and receives the plaintext. However, this only transforms the problem of encryption into the problem of transmitting the key. For example, you could distribute disks with the (very long) key in advance using a trusted messenger.

This is where quantum key distribution comes in. The key is distributed in the form of individual photons so that both the sender (Alice) and the receiver (Bob) can later use the same key. The encrypted message can then be transmitted over a normal channel. 

I describe here the BB84 protocol\sidenote{see also chapter 11.8.2 in \cite{Gerry_Knight_QO}, original in \cite{BB84}} by Charles Bennett and Gilles Brassard. We use four linear polarization states of light: horizontal ($\ket{h}$) and vertical ($\ket{v}$), diagonal ($\ket{+}$) and anti-diagonal ($\ket{-}$). They are not independent of each other, but $\ket{h}$ and $\ket{v}$ as well as $\ket{+}$ and $\ket{-}$ form a basis.

Alice sends photons to Bob and randomly chooses one of the four states for each photon. Bob is unaware of this, chooses one of the two bases at random, and measures the polarization state of the incoming photon in that basis, for example using a polarization beam splitter and two photodetectors. The bases could be selected using an appropriately rotated waveplate. When Alice and Bob have finished transmitting and measuring, Alice transmits the basis she has chosen over an open channel, not the polarization state. Bob compares this to his list and also tells Alice via an open channel if they have chosen the same basis. Both delete the other photons. But now they both have a list of polarization states that Bob measured in the same basis that Alice transmitted in. The polarization state is now a bit of a key.

Individual photons are essential. Otherwise, the eavesdropping Eve could intercept part of the beam and measure it herself, possibly even in both bases at the same time. Only if it is a single photon is it certain that the measurement destroys the state and it cannot be measured a second time.



\section{Ladder operators}


Let us look at the quantum mechanics of a photon. We have $n$ photons in our mode and an energy
\begin{equation}
    E_n = \left( n + \frac{1}{2} \right) \hbar \omega
\end{equation}
as in a harmonic oscillator. We have a ladder of equidistant states, bounded at the bottom by $E_0$, but open at the top, since $n$ can be arbitrarily large. In quantum mechanics it is convenient to use ladder operators for the creation ($\hat{a}^\dagger$) and annihilation ($\hat{a}$) of an energy quantum, i.e.
\begin{equation}
 \hat{a}^\dagger \ket{n} = \sqrt{n+1}\, \ket{n+1}  \quad \text{and} \quad
  \hat{a} \ket{n} = \sqrt{n}\, \ket{n-1} \quad .
\end{equation}
Useful properties are 
\begin{equation}
 \hat{a} \ket{0} = \ket{0}  \quad \text{and} \quad
  \hat{a}^\dagger  \hat{a} \ket{n} = n \ket{n} \quad .
\end{equation}


The electrical field of a single optical mode in a cavity can then be written \footcite[chap. 2.1 and 2.4]{Gerry_Knight_QO}\footcite[chap. 6.1]{Rand2016} using the vacuum field amplitude (eq. \ref{eq:8_evac})
\begin{equation}
\hat{\boldsymbol{E}}(z,t) = \boldsymbol{\hat{x}} \, E_{vac} \, \left(\hat{a} \, e^{i (k z - \omega t)} + \hat{a}^\dagger   \, e^{-i (k z - \omega t)} \right) 
\quad ,
\end{equation}
where $\boldsymbol{\hat{x}}$ is a unit vector defining the direction of polarization. The field in the cavity is a superposition of right and left propagating plane waves. The amplitude of each is related to the operators $\hat{a}$ and $\hat{a}^\dagger$, respectively.


\section{Beam splitter}

From the point of view of classical electrodynamics, a beam splitter seems to be a rather trivial device. But quantum optics will surprise you. Let us recall what electromagnetic waves do at a beam splitter: We shine in with a field $\bE_1$. Part of it is reflected (coefficient $r$), part is transmitted (coefficient $t$). The two output fields are
\begin{equation}
    \bE_2 = r \, \bE_1 \quad \text{and} \quad  \bE_3 = t \, \bE_1 \label{eq:8_bs_classic}
\end{equation}
where conservation of energy requires that 
\begin{equation}
    |r|^2 + |t|^2 = 1 \quad .
\end{equation}


In quantum optics we describe the modes for the input, reflected and transmitted beam by three ladder operators $\hat{a}_i$ with $i = 1,2,3$. Operators on different modes commute, i.e. have a commutator of zero. Only the pair of operators on the same mode have a commutator of one. As equations
\begin{align}
    \left[ \hat{a}_i, \hat{a}_j^\dagger  \right] = &  \delta_{ij} \\
    \left[ \hat{a}_i, \hat{a}_j  \right] = &  0 \\
    \left[ \hat{a}^\dagger_i, \hat{a}^\dagger_j  \right] = &  0  \quad .
\end{align}

Writing eq. \ref{eq:8_bs_classic} in the formalism of quantum optics gives
\begin{equation}
    \hat{a}_2 = r \, \hat{a}_1 \quad \text{and} \quad  \hat{a}_3 = t \, \hat{a}_1  \quad . \label{eq:8_bs_QO1}
\end{equation}
The problem is that these definitions do not satisfy the commutators. We can check several combinations and all of them give the required result of zero only if we set either $r=0$ or $t=0$, i.e. if we remove the beam splitter
\begin{align}
    \left[ \hat{a}_2, \hat{a}_2^\dagger  \right] = &  |r|^2 \left[ \hat{a}_1, \hat{a}_1^\dagger  \right] =  |r|^2  \neq 1 \\
    \left[ \hat{a}_3, \hat{a}_3^\dagger  \right] = &  |t|^2 \left[ \hat{a}_1, \hat{a}_1^\dagger  \right] =  |t|^2  \neq 1 \\
    \left[ \hat{a}_2, \hat{a}_3^\dagger  \right] = &  rt^\star \left[ \hat{a}_1, \hat{a}_1^\dagger  \right] =   rt^\star  \neq 0  \quad .
\end{align}
So in quantum optics there seems to be something more to the beam splitter than in classical electrodynamics. The point is the fourth side of the beam splitting cube (which we label $i=0$ for convenience). Classically, we assumed that no light would enter here, so we set $\bE_0 = 0$ and ignored it in our calculations. But in quantum optics we have to take into account that $|\bE_0| = E_{vac}$. The vacuum fluctuations are shining into the empty port of the beam splitter.

We expand eq.  \ref{eq:8_bs_QO1} to take the fourth port into account. In matrix form
\begin{equation}
    \begin{pmatrix}
        \hat{a}_2 \\ \hat{a}_3 \\
    \end{pmatrix}
    = 
    \begin{pmatrix}
       t' & r \\ r' & t
    \end{pmatrix}
    \begin{pmatrix}
        \hat{a}_0 \\ \hat{a}_1 \\
    \end{pmatrix} \quad .
\end{equation}
This definition satisfies all commutator relations when the reflection and transmission coefficients satisfy energy conservation ($|r|^2+ |t|^2 = 1$), reciprocity ($|r'| = |r|$, $|t'| = |t|$), and a phase condition as discussed with the Michelson interferometer. For a single dielectric layer, the transmitted and reflected beams are 90 degrees out of phase. A 50:50 beamsplitter is thus
\begin{equation}
\frac{1}{\sqrt{2}}
    \begin{pmatrix}
       1 & i \\ i & 1
    \end{pmatrix} \quad .
\end{equation}


\section{A single photon at a 50:50 beam splitter}

As an example, let us examine what happens to a single photon in a 50:50 beam splitter. I label the eigenstates
with $n$ photons in beam $i$ as $\ket{n}_i$. If nothing (a vacuum) shines into a beam splitter, nothing comes out, i.e. trivially
\begin{equation}
    \ket{0}_0 \,  \ket{0}_1  \xrightarrow{BS}  \ket{0}_2 \,  \ket{0}_3 \quad .
\end{equation}
A single photon at port 1 is written by the creation operator acting on the vacuum state
\begin{equation}
    \ket{0}_0 \,  \ket{1}_1 = \hat{a}^\dagger_1 \ket{0}_0 \,  \ket{0}_1 
\end{equation}
and $\hat{a}^\dagger_1$ can be written with the port 2 and 3 operators of a 50:50 beam splitter
\begin{equation}
    \hat{a}^\dagger_1 = \frac{1}{\sqrt{2}} \left( i \hat{a}^\dagger_2 + \hat{a}^\dagger_3 \right)
\end{equation}
so that 
\begin{equation}
    \ket{0}_0 \,  \ket{1}_1  \xrightarrow{BS} \frac{1}{\sqrt{2}} \left( i   \ket{1}_2 \,  \ket{0}_3 +   \ket{0}_2 \,  \ket{1}_3 \right) \quad .
\end{equation}
A single photon entering port 1 together with a vacuum at port 0 leaves either port 2 or port 3, but not both! The final state is a \emph{entangled state}\sidenote{dt: verschränkter Zustand}, since it cannot be written as a product of states of the form $\ket{n}_2 \ket{m}_3$.




\section{Hong-Ou-Mandel experiment: two photons on a beam splitter}

\begin{marginfigure}
    \inputtikz{\currfiledir HOM}
    \caption{Two identical photons entering a beam splitter leave together (top), because the two processes below cannot be distinguished and interfere destructively.}
\end{marginfigure}


What we can do with one photon, we can do with two! Let us shine a photon on each input port of the beam splitter, i.e. start with the state $\ket{1}_0 \, \ket{1}_1 $. We can create it from vacuum with two creation operators
\begin{equation}
    \ket{1}_0 \,  \ket{1}_1 = \hat{a}^\dagger_1 \hat{a}^\dagger_0 \ket{0}_0 \,  \ket{0}_1 
\end{equation}
and write additionally to $ \hat{a}^\dagger_1$ as above also  $\hat{a}^\dagger_0$ in terms of the port 2 and 3 operators 
\begin{equation}
    \hat{a}^\dagger_0 = \frac{1}{\sqrt{2}} \left(  \hat{a}^\dagger_2 + i \hat{a}^\dagger_3 \right)
\end{equation}
where only the position of the $i$ has swapped. Everything together is
\begin{align}
    \ket{1}_0 \,  \ket{1}_1  \xrightarrow{BS} & \frac{1}{2} 
    \left( i \hat{a}^\dagger_2 +  \hat{a}^\dagger_3\right) 
     \left( \hat{a}^\dagger_2 + i \hat{a}^\dagger_3\right) 
      \ket{0}_2 \,  \ket{0}_3 \\
      & = \frac{i}{2} 
      \left(  \hat{a}^\dagger_2  \hat{a}^\dagger_2 +  \hat{a}^\dagger_3 \hat{a}^\dagger_3\right) 
        \ket{0}_2 \,  \ket{0}_3 \\
        & = \frac{i}{\sqrt{2}} 
        \left(    \ket{2}_2 \,  \ket{0}_3 +  \ket{0}_2 \,  \ket{2}_3 \right)  \quad .
\end{align}
When multiplying out the two brackets, the cross terms $\hat{a}^\dagger_2  \hat{a}^\dagger_3$ cancel out and only the symmetric terms remain.  Two photons entering at different ports of the beam splitter will exit through the same port! Either both at port 2 or both at port 3. This is a quantum interference effect. The case where both photons are transmitted interferes destructively with the case where both photons are reflected. Since the photons are indistinguishable, we cannot separate the two cases and must add the probability amplitudes before taking the square modulus to get the power. This was demonstrated in 1987 by Hong, Ou, and Mandel\footcite{HOM87}.



The word 'indistinguishable' is important here. The two photons must agree in every conceivable way. They must arrive at the beam splitter at the same time. They must have the same spectrum/color, the same polarization state, and the same mode/wavefront or beam profile. As soon as one photon is a little different from the other, we could distinguish the 'both reflected' case from the 'both transmitted' case.
 
The Hong-Ou-Mandel experiment shows that photons are bosons. Bosons want to be in the same state. The two photons want to leave the beam splitter together. This is in stark contrast to electrons, which are fermions. Fermions want to be in different states. A beam splitter for electron waves would favor the $\ket{1}_2\ket{1}_3$ result over a $\ket{2}_2\ket{0}_3$ result.



\section{Down-conversion source of entangled photons}

Experiments in quantum optics require a single-photon source or a source of single entangled photon pairs. A single-photon source such as an atom, a molecule or a quantum dot can be studied with current technology even in an undergraduate lab. But if it should be the source for the following actual experiment, it is a bit too complicated. A technically simpler device is a source of entangled photon pairs (but not 'single' pairs). The process is called down-conversion.

In nonlinear optics, there are processes that split a photon into two. The energy of the incoming photon is split into two parts, leaving as two photons. This is called down conversion. There is also the reverse process where the energy of two photons is combined into one. This is called sum frequency generation, since the optical frequency of the outgoing photon is the sum of the frequencies of the incoming photons.


Both processes require that the polarization $P$ induced by a field $E$ is to some extent quadratically dependent on the field. In chapter \ref{chap:dielectrics} we wrote 
\begin{equation}
    P(t) =  \chi \epsilon_0 E(t)  \quad .
\end{equation}
Now we see this as first order of a Taylor expansion and write 
\begin{equation}
    P(t) =  \epsilon_0 \left( \chi^{(1)} E(t) +  \chi^{(2)} E^2(t)   +  \chi^{(3)} E^3(t)  + \dots \right) \quad .
\end{equation}
The expression with $\chi^{(2)}$ is important for us. If 
\begin{equation}
    E(t) = E_1 e^{i \omega_1 t} + E_2 e^{i \omega_2 t}
\end{equation}
then $P(t)$ will contain frequency components at $\omega_1$ and $\omega_2$ due to the $\chi^{(1)}$ term and at 0, $2\omega_1$, $2\omega_2$ and $\omega_1 \pm \omega_2$ due to the $\chi^{(2)}$ term. The latter is sum and difference frequency generation.

In a down-conversion source, a blue laser beam hits a material with a rather high value of $\chi^{(2)}$: BBO  (Beta Barium Borate).The energy of the 400~nm photon is split into two photons of 800~nm wavelength. This conserves energy. Conservation of momentum would be easy if the index of refraction at both wavelengths were identical. However, this is not the case, as we saw in chapter \ref{chap:dielectrics}. This is where birefringence comes in. The ordinary and the extraordinary ray have slightly different refractive indices. In the chosen material, this difference helps to compensate for the difference due to dispersion. As a result, one of the 800~nm photons leaves as the ordinary beam and the other as the extraordinary beam. Both photons are polarized orthogonal to each other. By carefully aligning the source, one can generate polarization entangled photons, i.e. a state in which one does not know how each photon is polarized, except that one knows they are orthogonal to each other.

%--------------------
\printbibliography[segment=\therefsegment,heading=subbibliography]


%-------
\renewcommand{\kapitelname}{Appendix\ }

\addcontentsline{toc}{part}{Appendix} 
\appendix
\appendixpage
\renewcommand{\lastmod}{September 18, 2023}
\renewcommand{\chapterauthors}{Markus Lippitz}

\chapter{Fourier transformation}

\section{Overview}

It is useful and helpful to have an intuitive approach to the Fourier transform. The bottom line is that in experimental physics one rarely needs to actually calculate a Fourier transform. Very often it is sufficient to know a few frequently occurring Fourier pairs and to combine them with simple rules. This is what I want to present here. A very nice and much more detailed presentation can be found in \cite{Butz2015}. I will follow his notation here.

Before we get to Fourier pairs, however, we need to lay down some foundations.

\section{Fourier series: a periodic function and its Fourier coefficients}

We first consider everything here in one dimension in time or frequency space with the variables $t$ and $\omega = 2 \pi \nu$. Let the function $f(t)$ be periodic in time with period $T$, i.e. 
\begin{equation}
 f(t) = f (t + T) \quad .
\end{equation}
Then this can be written as a Fourier series
\begin{equation}
 f(t) = \sum_{k=-\infty}^{\infty} \, C_k \, e^{i \, \omega_k \, t}
 \quad \text{with} \quad \omega_k = \frac{2 \pi \, k}{T}
\end{equation}
and the Fourier coefficients
\begin{equation}
 C_k = \frac{1}{T} \, \int_{-T/2}^{T/2} \, f(t) \, \, e^{-i \, \omega_k \, t} \, dt \quad .
\end{equation}
Note the negative sign in the exponential function in contrast to the equation before. For reel-valued functions $f(t)$, 'opposite' $C_k$ are conjugate-complex, so $C_k = C_{-k}^\star$. For $k<0$ the frequencies $\omega_k$ are negative, but this is not a problem.\sidenote{One could alternatively require $k\ge 0$ and apply a $\sin$ and $\cos$ series.} Thus, the zeroth coefficient $C_0$ is just the time average of the function $f(t)$.



\section{An arbitrary function and its Fourier transform}

Now we remove the restriction to periodic functions $f(t)$ by letting the period $T$ go to infinity. This turns the sum into an integral and the discrete $\omega_k$ become continuous. Thus
\begin{align}
 F(\omega) = & \int_{-\infty}^{+\infty} \, f(t) \, e^{- i \omega\, t} \, dt \\
 f(t) = & \frac{1}{2 \pi } \int_{-\infty}^{+\infty} \, F(\omega) \, e^{+ i \omega\, t} \, d\omega \quad .
\end{align}
Here, the first equation is the forward transformation (minus sign in the exponent), and the second is the reverse transformation (plus sign in the exponent). The symmetry is broken by the $2 \pi$. But this is necessary if one wants to keep $F(\omega = 0)$ as mean\sidenote{$F( 0) = \int f(t) \, dt$ without $1/T$ in front of it is meant here by Butz as mean!}. Alternatively, we could formulate all this with $\nu$ instead of $\omega$, but then we would have a $2 \pi$ in many more places, though not before the integral.





\section{Sidenote: Delta Function}

The delta function can be written as
\begin{equation}
  \delta(x) = \lim_{a \rightarrow 0} f_a(x) \quad
   \text{with} \quad
    f_a(x) = \left\{ \begin{matrix}
    a & \text{if } |x| < \frac{1}{2a} \\
    0 & \text{other}
    \end{matrix}
    \right.
\end{equation}
or as
\begin{equation}
\delta(x) = \frac{1}{2 \pi}  \int_{-\infty}^{+\infty} \, e^{+ i\, x \, y} \, dy \quad .
\end{equation}
An important property is that the delta function selects a value, i.e. 
\begin{equation}
 \int_{-\infty}^{+\infty} \, \delta(x) \, f(x) \, dx = f(0) \quad .
\end{equation}


\section{Important Fourier pairs}

It is very often sufficient to know the following pairs of functions and their Fourier transforms. I write them here, following Butz, as pairs in $t$ and $\omega$ (not $\nu = \omega / (2 \pi)$). In the same way, one could have written pairs in $x$ and $k$. The important question is whether a $2 \pi$ appears in the exponential function of the plane wave or not. So
\begin{equation}
e^{i \omega t} \quad \text{and} \quad e^{i k x} \quad \text{, but} \quad 
e^{i 2 \pi \nu t} \quad .
\end{equation}

Further, I follow here the convention made above about the asymmetric distribution of the $2 \pi$ between forward and reverse transformations. If you distribute them differently, then of course the prefactors change. A good overview of many more Fourier pairs in various '$2 \pi$' conventions can be found in the English Wikipedia under 'Fourier transform'. In their nomenclature, the Butz convention used here is 'non-unitary, angular frequency'.

\paragraph{constant and delta function} $f(t) = a$ becomes $F(\omega) = a \, 2 \pi \, \delta(\omega)$ and $f(t) = a \, \delta(t)$ becomes $F(\omega) = a $. This is again the asymmetric $2 \pi$.


\paragraph{rectangle and sinc} The rectangle function of width $b$ becomes a sinc\sidenote{sometimes $\text{sinc}(x) = \sin (\pi x) / (\pi x)$ is defined, especially when $\nu$ and not $\omega$ is used as conjugate variable.}, the sinus cardinalis. So from
\begin{equation}
 f(t) = \text{rect} _b (t) = \left\{ 
 \begin{array}{ll}
 1 & \text{for} \quad |t| < b/2 \\
 0 & \text{other} \\
 \end{array}
 \right.
\end{equation}
we get
\begin{equation}
F(\omega) = b \, \frac{\sin \omega b / 2}{\omega b /2} = b \, \text{sinc}( \omega b /2) \quad .
\end{equation}



\paragraph{Gaussian} The Gaussian function is preserved under Fourier transform. Its width changes into the reciprocal value. So from a Gauss function of area one
\begin{equation}
 f(t) = \frac{1}{\sigma \sqrt{2 \pi}} \, e^{- \frac{1}{2} \left( \frac{t}{\sigma} \right)^2}
\end{equation}
we get
\begin{equation}
 F(\omega) = e^{- \frac{1}{2} \left( \sigma \, \omega \right) ^2 } \quad .
\end{equation}



\paragraph{(two-sided) exponential decay and Lorentz curve} From a curve decaying exponentially at both positive and negative times
\begin{equation}
 f(t) = e^{- |t| / \tau}
\end{equation}
we obtain the Lorentz curve
\begin{equation}
 F(\omega) = \frac{2 \tau}{1 + \omega^2 \, \tau^2} \quad .
\end{equation}


\paragraph{one-sided exponential decay} As a side note, here  the one-sided exponential decay
\begin{equation}
 f(t) = \left\{ \begin{array}{ll}
e^{- \lambda t } & \text{for} \quad t > 0 \\
 0 & \text{other} \\
 \end{array}
 \right. \quad .
\end{equation}
It will become
\begin{equation}
 F(\omega) = \frac{1}{\lambda + i \, \omega}
\end{equation}
and it is therefore complex-valued. Its magnitude squared is again a Lorentz function
\begin{equation}
| F(\omega)|^2 = \frac{1}{\lambda^2 + \omega^2}
\end{equation}
and the phase is $\phi = - \omega / \lambda$.


\paragraph{One-dimensional point lattice} An equidistant chain of points or delta functions remains an equidistant chain under Fourier transform. The distances take  the reciprocal value. So from
\begin{equation}
 f(t) = \sum_n \, \delta (t - \delta t \, n)
\end{equation}
we get
\begin{equation}
 F(\omega) = \frac{2 \pi}{\delta t} \, \sum_n \, \delta \left(\omega - n\frac{2 \pi}{\Delta t} \right). \quad .
\end{equation}


\paragraph{Three-dimensional cubic lattice} A three-dimensional primitive cubic lattice of side length $a$ makes the transitions to a primitive cubic lattice of side length $2 \pi/a$. A face-centered cubic lattice with lattice constant $a$ of conventional unit cell is converted  to a space-centered cubic lattice with lattice constant $4 \pi / a$ and vice versa. 


\section{Theorems and properties of the Fourier transform}

In addition to the Fourier pairs, we need a few properties of the Fourier transform. In the following, let $f(t)$ and $F(\omega)$ be Fourier conjugates and likewise $g$ and $G$.

\paragraph{linearity} The Fourier transform is linear
\begin{equation}
a \, f(t) + b \, g(t) \quad \leftrightarrow \quad 
a \, F(\omega) + b \, G(\omega)  \quad .
\end{equation}

\paragraph{shift} A shift in time implies a modulation in frequency and vice versa.
\begin{align}
 f(t - a) & \quad \leftrightarrow \quad 
F(\omega) \, e^{-i \omega a} \\
 f(t) \, \, e^{-i \omega_0 t} & \quad \leftrightarrow \quad 
F(\omega + \omega_0)   \quad .
\end{align}

\paragraph{scaling}  
\begin{equation}
 f( a \, t) \quad \leftrightarrow \quad 
\frac{1}{|a|} \, F \left( \frac{\omega}{a} \right)   \quad .
\end{equation}


\paragraph{convolution and multiplication} Convolution is converted into a product, and vice versa
\begin{equation}
 f(t) \otimes g(t) = \int f(\zeta) g(t- \zeta) d\zeta 
 \quad \leftrightarrow \quad 
 F(\omega) \, G(\omega)
\end{equation}
and
\begin{equation}
 f(t) \, g(t) 
 \quad \leftrightarrow \quad 
\frac{1}{2 \pi} \, F(\omega) \otimes G(\omega) \quad .
\end{equation}

\paragraph{Parseval's Theorem} The total power is the same in both time and frequency domain
\begin{equation}
 \int |f(t) |^2 \, dt = \frac{1}{2 \pi} \, \int | F (\omega ) | ^2 \, d\omega
\end{equation}

\paragraph{time derivatives}
\begin{equation}
 \frac{d \, f(t)}{dt} 
 \quad \leftrightarrow \quad 
i \omega \, F(\omega)  \quad .
\end{equation}


\section{Example: Diffraction at a double slit}

As an example, we consider the Fourier transform of a double slit, which describes its diffraction pattern. The slits have a width $b$ and a center distance $d$. Thus the slit is described by a convolution of the rectangular function with two delta functions at the distance $d$
\begin{equation}
f(x) = \text{rect} _b (x) \, \otimes \, \left( \delta (x - d/2) + \delta (x + d/2) \right) \quad .
\end{equation}
The Fourier transform of the rectangular function is the $\text{sinc}$, that of the delta functions a constant. However, the shift in position causes a modulation in $k$-space. Thus, the sum of the two delta functions becomes 
\begin{equation}
\mathcal{FT}\left\{ \delta (x - d/2) + \delta (x + d/2) \right\} =
e^{-i k d/2} + e^{+i k d/2} = 2 \cos ( k d/2) \quad .
\end{equation}
The convolution with the rectangular function passes into a multiplication with the $\text{sinc}$. Together we get
\begin{equation}
\mathcal{FT}\left\{ f(x) \right\} = b \frac{\sin (k b/2) }{kb/2} \, 2 \cos ( k d/2) = \frac{4}{k} \, \sin (k b/2) \, \cos ( k d/2)  \quad .
\end{equation}
The intensity in direction $k$ is then the squared magnitude  of this.




\begin{questions}
  \item \emph{Temporal shift}
    Sketch the amplitude and phase of the FT of a temporal square  pulse pulse centred on time zero!    What changes if the pulse is shifted to positive times?

    \item \emph{Pulse sequence} You wonder what the Fourier transform (magnitude squared) of an infinite sequence of square pulses looks like and start searching for it on the internet. Your fellow student replies that you can "see" it immediately.
    Sketch the Fourier transform!
    Explain why you could derive it directly or why you should "see" it!

    \item \emph{Light pulse}
    Think of a "light pulse" as a mathematical construction of an infinitely long cosine oscillation corresponding to the frequency of light. The "pulse" is obtained by multiplying the wave by a time-limited Gaussian pulse envelope (e.g. half-width of 10 light oscillations).
    Sketch the construction of the Fourier transform in the spectral domain.
\end{questions}




\section{Two-dimensional Fourier transformation}

We can extend the definition of the Fourier transform to two and more dimensions. The conjugated variables are $(x,y)$ and $(k_x, k_y)$ instead of $t$ and $\omega$. The wave vector $k_i = 2\pi / \lambda_i$ contains the factor $2\pi$ as in the angular frequency $\omega$. We define
\begin{align}
  F(k_x, k_y) = & \iint_{-\infty}^{+\infty} \, f(x,y) \, e^{- i (k_x \, x + k_y \, y )} \, dx \, dy \\
  f(x,y) = & \frac{1}{(2 \pi )^2} \iint_{-\infty}^{+\infty} \, F(k_x, k_y) \,\, e^{+ i (k_x \, x + k_y \, y )}  \, dk_x \,dk_y \quad .
 \end{align}

When we can separate the function $f(x,y)$ into a product of one-dimensional functions, then  the Fourier transform is simply the product of the individual Fourier transforms
\begin{equation}
  f(x,y) = g(x) \cdot h(y) \quad \leftrightarrow \quad 
  F(k_x, k_y) = G(k_x) \cdot H(k_y) \quad .
\end{equation}

A rectangle of size $a \times b$ is transformed into a product of sinc functions
\begin{align}
  (x,y) = & \text{rect} _a (x) \cdot \text{rect} _b (y) \\
  \leftrightarrow \quad  F(k_x, k_y) = & a b \, \text{sinc}( k_x a /2) \, \text{sinc}( k_y b /2) \quad .
\end{align}


A special case of this is the rotational symmetric two-dimensional Gaussian function
\begin{equation}
  f(x,y) = 
  \frac{1}{2 \pi \sigma^2} \, e^{-  \frac{x^2 + y^2}{2 \sigma^2} }
  \quad \leftrightarrow \quad 
  F(k_x, k_y) = e^{- \frac{\sigma^2 }{2} \left(k_x^2 + k_y^2 \right)  } \quad .
\end{equation}

One important function can not be separated into a product of one-dimensional functions: a disc of radius $a$
    \begin{equation}
    f(x,y)  = \left\{ 
    \begin{array}{ll}
    1 & \text{for} \quad x^2+y^2 < a \\
    0 & \text{other} \\
    \end{array}
    \right.
   \end{equation}
is transformed into 
\begin{equation}
  F(k_x, k_y) = a \, \frac{J_1(\pi \, a \, \rho )}{\rho}
  \quad \text{width} \quad \rho = \sqrt{k_x^2 + k_y^2}
\end{equation}
and the (cylindrical) Bessel function of the first kind $J_1(x)$
\begin{equation}
  J_1(x) = \frac{1}{\pi} \int_0^\pi \cos (\tau - x \sin \tau) \,d\tau \quad ,
\end{equation}
which is the cylindrical analogue of a sinc function.





%--------------------
\printbibliography[segment=\therefsegment,heading=subbibliography]

\renewcommand{\lastmod}{September 18, 2023}
\renewcommand{\chapterauthors}{Markus Lippitz}

\chapter{Numerical Fourier Transformation}



\section{Discrete FT: a periodic sequence of values}


In particular, if one collects and evaluates measurement data with a computer, then one does not know the measured function $f(t)$ on a continuous axis $t$, but only at discrete times $t_k = k \, \delta t$, nor does one know the function from $t = - \infty$ to $t = + \infty$. So we have only a finite sequence of numbers $f_k$ as a starting point.
Because we do not know the sequence of numbers outside the measured interval we make the assumption that it is periodic. With $N$ measured values the period is $T = N \Delta t$. For simplicity, we also define $f_k = f_{k + N}$ and thus $f_{-k} = f_{N - k}$ with $k= 0, 1, \dots, N-1$. Thus the Fourier transform becomes\sidenote{see \cite{Butz2015} chap. 4, \cite{Horowitz_Hill}, chap. 1.08, 7.20, 15.18}
\begin{equation}
  F_j =  \frac{1}{N} \, \sum_{k=0}^{N-1} \, f_k \, e^{- k \, j \, 2 \pi i / N } 
 \end{equation}
and its inverse transform
 \begin{equation}
 f_k =   \sum_{j=0}^{N-1} \, F_j \, e^{+ k \,  j \, 2 \pi i / N } \quad .
 \end{equation}
The definition is again such that $F_0$ corresponds to the mean. Because of $f_{-k} = f_{N - k}$, the positive frequencies are in the first half of $F_j$ as the frequency increases. After that come the negative frequencies, starting at the 'most negative' frequency and increasing to the last frequency before zero. So the maximum frequency that can be represented is the Nyquist (angular) frequency
\begin{equation}
\Omega_\text{Nyquist} = \frac{\pi}{\delta t} \quad .
\end{equation}
%
This frequency is such that we take two samples per period of the
oscillation. Faster oscillations or fewer samples per period cannot be
represented. Even with $f_\text{Nyquist}$ the imaginary part is always zero, because we always sample the sine at the 
zero crossing.


% \begin{jllisting}
%   using Plots
%   x = range(0, 2 * pi; length=100)
%   plot(x, sin.(x); label="ein Sinus")
% \end{jllisting}


\section{FFTW}

The most used package for numerical Fourier transform is probably FFTW\sidenote{\url{https://www.fftw.org/}}. You have to pay attention to the details of the definition. In particular, the prefactors may differ between different packages. In FFTW, the prefactor $1/N$ changes from the forward to the backward transformation, i.e.
%
\begin{equation}
 F_j =   \sum_{k=0}^{N-1} \, f_k \, e^{- k \, j \, 2 \pi i / N } 
\end{equation}
%
and the inverse Fourier transform
%
\begin{equation}
f_k =  \frac{1}{N} \, \sum_{j=0}^{N-1} \, F_j \, e^{+ k \,  j \, 2 \pi i / N } \quad .
\end{equation}
In equations, I (and Butz) use mathematical indices (starting from zero).
Some programming languages count from one (e.g., Julia).

One helpful thing of FFTW is that is supplies also a frequency axis. As mentioned above, first come the positive frequencies, starting from zero to the maximum, then the most negative frequency, again rising until just before zero. 
Depending wether the number of samples $N$ is  even or odd, it is a little bit of a hassle 
to calculate the respective frequencies, but FFTW   does this for us:
\begin{jllisting}
  fftfreq(5) # gives [0.0, 0.2, 0.4, -0.4, -0.2]
  fftfreq(6) # gives [0.0, 0.166, 0.333, -0.5, -0.333, -0.1.66]
\end{jllisting}

\begin{questions}
  \item Try yourself the FFT in a language of your choice. The FFT of, say, $[1 1 1 1]$ should give something like $[4 0 0 0 ]$. 
  \item The inverse FFT is IFFT. Check that it inverts and test how the pre-factors are distributed.
\end{questions}



\subsection{Wrapping \& fftshift}

Now lets look at the Fourier transform of a cosine. We evaluate the cosine at 8 points:
\begin{align}
  x_n = & n \frac{2 \pi}{8} \quad \text{with} \quad n = 0 \dots 7 \\
  f_n = & \cos x_n \\
  F = & \mathcal{FT} (f) \quad .
\end{align}
We find that only $F_1$ and $F_7$ are different from zero and have the same, real value.
Two values must be different from zero because
\begin{equation}
 \cos(x) = \frac{1}{2} \left(e^{i x} + e^{-i x} \right) \quad .
\end{equation}
In general, for real  values $f_n$ we have
\begin{equation}
F_{N-j} = F_j^\star \quad .
\end{equation}



The position of these two non-zero values is a
consequence of the definition of $F_k$: first come all positive
frequencies and then all negative. For a nicer representation it is often better if the frequency zero is not the first element
but in the middle between the positive and negative frequencies.
frequencies. This we get by  \jlinl{fftshift} or backwards by \jlinl{ifftshift}.  

\begin{questions}
  \item Convince yourself that you understand why it is element 1 and 7 that differs from zero in the example above.
  \item Replace the cosine with a sine in this example and
  explain the result.
\end{questions}
  



\section{Sampling theorem}

We need at least two samples per period to describe a function by its
   Fourier coefficients. The frequencies must be
   below the Nyquist frequency $f_\text{Nyquist}$
%
\begin{equation}
f_\text{Nyquist} = \frac{1}{2 \Delta t} \quad .
\end{equation}
%
The \emph{sampling theorem} states that this is then also sufficient, i.e., we do not lose any detail by sampling.
Let $f(t)$ be a bandwidth-limited function, i.e.
$F(\omega)$ is different from zero only in the interval $|\omega| \le \Omega_\text{Nyquist}$. Then the sampling theorem\sidenote{for a proof see \cite{Butz2015}, chap. 4.4} applies and gives 
%
\begin{equation}
f(t) \overset{!}{=} \sum_{k=-\infty}^{\infty} \, f( k \Delta t) \, \text{sinc} \left( \Omega_\text{Nyquist} \cdot [t - k \Delta t] \right) \quad .
\end{equation}
So it is enough to sample $f$ all $\Delta t$. At the
times in between, $f$ is completely described by the (infinitely long) sum of the
neighbouring values times the sinc.

In measurement technology, therefore, all we need to do is ensure, for example by means of an electrical filter, that all the frequencies of a signal are below
$\Omega_\text{Nyquist}$, and then our digital acquisition of the signal will be
is identical to the signal itself.
However, if we sample too infrequently, or if there are higher frequencies present, then these too high frequency components will be reflected at the 
Nyquist frequency and end up at seemingly lower frequencies. This 'aliasing' distorts the signal.
 



\section{Zero padding}


We began with a repeating pattern of numerical values and their
Fourier transform. We always picked the length of the sequence in the examples to match an integer multiple of the period.
But of course, this isn't feasible in reality. We lack accurate knowledge of the signal's duration.
Or sometimes, multiple signals with varying frequencies are important.

The problem is then a truncation error, which leads to artefacts in the
Fourier transform.  Fig. \ref{fig:1_clipping} shows an example. 12 data points of a cosine with period 8 are sampled. The FFT assumes periodic continuation (thick) which is not the 'true' signal (thin). In this case, the FFT of the data is far from a peak at the original frequencies. The real part is even spectrally constant (see below Fig. \ref{fig:1_zeropadding})

\begin{marginfigure}
  \inputtikz{\currfiledir clipping_artefact}
  \caption{Clipping a cosine after 1.5 periods }
  \label{fig:1_clipping}
\end{marginfigure}



The way out is \emph{zero-padding}. Let our actual
measured signal sequence $f(t)$, which we know in the interval $[-T, T]$.
Now we pretend that we measured instead
\begin{equation}
g(t) = f(t) \cdot w(t)
\end{equation}
with the window function $w(t)$
\begin{equation}
w(t) = 1 \quad \text{for} \quad -T < t < T \quad \text{other} = 0 \quad .
\end{equation}
Thus we can `measure' $g(t)$ over arbitrarily long times, because it is
is quasi always zero. But the Fourier transform is
\begin{equation}
G(\omega) = F(\omega) \otimes W(\omega)
\end{equation}
with
\begin{equation}
W(\omega)= 2T \, \frac{\sin \omega T}{\omega T} = 2T \, \text{sinc}( \omega T). \quad .
\end{equation}

 
So we extend our data set on both sides with zeros.
The effect is that we convolve the actual Fourier transform of our
data set with a $\text{sinc}$ whose characteristic width is determined by the actual
measurement duration. The
frequency resolution does not increase. Rather, a kind of
interpolation in Fourier space occurs, which just eliminates the artefacts of the
truncation error.  


We consider the same data set as above, only we 'extend' it to 10 times the length. This means that the clipping error has less
influence and the peak is always at 1 Hz in frequency space. But this does not give
more resolution, of course. Peaks that are close to each other cannot be
separated by zero-padding, only the position of a peak can be be determined better .  

\begin{marginfigure}
  \inputtikz{\currfiledir zeropadding}
  \caption{Zeropadding (line) approaches better the real spectrum (filled symbols) compared to the clipped FT (open symbols).}
  \label{fig:1_zeropadding}
\end{marginfigure}








\section{Windowing}



The oscillations in the spectrum in the last example are still
artefacts. Actually, one would expect two delta functions at $\pm 1$Hz. They are a consequence of the rectangular window $w(t)$, which
leads to the sinc in frequency space. The square-wave window is natural in the sense that we always start and stop measuring. Other window functions\sidenote{\url{https://en.wikipedia.org/wiki/Window_function}}, however, may be better.They differ the width of the peak and the steepness of the slopes. Unfortunately one must trade
one against the other. Interesting parameters are
the width of the central peak in frequency space, measured as a
-3dB bandwidth, as well as the sideband suppression in \sidenote{dB = decibel = $10 \log_10 x$} dB or its
drop in dB/octave.


Typical window functions are (with $|x| = |t/T| < 1/2$ )
\begin{align}
\text{cosine} &  = \cos \pi x \\
\text{triangle} = & 1 - 2 |x| \\
\text{Hanning} = & \cos^2 \pi x \\
 \text{Hamming} = & a + (1-a)\cos^2 \pi x  \\
 \text{Gauss} = & \exp \left( - \frac{1}{2} \frac{x^2}{\sigma^2} \right) \\
\text{Kaiser-Bessel} =  & \frac{I_0(\pi \alpha \sqrt{1-4 x^2})}{I_0(\pi \alpha)} 
\end{align}
with the modified Bessel function $I_0$.




With a window, the measured values are reduced, but the Fourier
transform is smoother, because the transition to the
zero padding becomes smoother. This makes it possible to recognize in the example the peaks at $\pm 1$Hz
even with very few sampled points. 

\begin{marginfigure}
  \inputtikz{\currfiledir zeropadding_window}
  \caption{Zeropadding after windowing (thick) removes the fringes of the un-windowed data (thin) and approaches the true spectrum (solid symbols).}
  \label{fig:1_zeropadding_window}
\end{marginfigure}






We consider as example\sidenote{from \cite{Butz2015}, chapter 3.10} a sum of 6 cosine functions with partly very different amplitudes $A_l$ and frequencies $f_l$:
\begin{align}
  f(t) = & \cos \omega t + 10^{-2} \cos 1.15 \omega t  + 10^{-3} \cos 1.25 \omega t \\
   & + 10^{-3} \cos 2 \omega t  + 10^{-4} \cos 2.75 \omega t + 10^{-5} \cos 3 \omega t \nonumber
\end{align}
We sample 256 data points at intervals of $\Delta t = 1/8$, i.e. only
$8/3 \approx 3$ data points per oscillation of the highest occurring frequency, which is 5 orders of magnitude weaker than the lowest frequency.
Nevertheless, this peak can be found with a suitable window and
zero-padding.  



\begin{marginfigure}
  \inputtikz{\currfiledir example_rect}

  \inputtikz{\currfiledir example_hanning}
  
  \caption{Without windowing (top), only the main signal component is recovered. A Hanning window (bottom) allows to find even signals $10^{-5}$ below the main component.}
  \label{fig:1_example_windowing}
\end{marginfigure}



%--------------------
\printbibliography[segment=\therefsegment,heading=subbibliography]

%
\
%-------
%
%%\nocite{*}
 
\printbibliography



\end{document}
