\renewcommand{\lastmod}{September 18, 2023}
\renewcommand{\chapterauthors}{Markus Lippitz}

\chapter{Coherence}

\section{Overview}

\ziel{Ich kann ein interessantes Experiment zu diesem Thema  \wort{entwerfen}, \wort{aufbauen} und \wort{erklären}.}


\section{Dielektrische Filter}
\ziel{Ich kann die Vielstrahl-Interferenz in einem planaren Schichtsystem numerisch \wort{berechnen}.}

\begin{itemize}
\item T- und S-Matrix
\item Aufbau und Design von dielektrischen Filter
\item distributed Bragg reflectors (DBR)
\end{itemize}

\lit{Saleh/Teich  Kap. 7, Hecht Kap. 9}


%-----------------------------------------------------------------------------
\section{Kohärenz}
\ziel{Ich kann räumliche und zeitliche Kohärenz erklären und Methoden zu deren experimentellen Bestimmung \wort{beschreiben}.}


\begin{itemize}
\item Überlagerung von Wellen verschiedener Frequenz  
\item Interferenz bei ausgedehnten Lichtquellen 
\item Korrelationsfunktion erster und zweiter Ordnung (Feld und Intensität)
\item Wiener-Khinchin(-Einstein)-Theorem 
\end{itemize}

\lit{Saleh/Teich  Kap. 11, Hecht Kap. 12}



%--------------------
\printbibliography[segment=\therefsegment,heading=subbibliography]
