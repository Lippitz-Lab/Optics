\renewcommand{\lastmod}{\ \ }
\renewcommand{\chapterauthors}{\ \ }

\chapter*{Preface}

These are the lecture notes for my lecture on optics. The lecture is aimed at students in the third year of the bachelor programme. It follows the idea of \cite{SalehTeich1991}: we start with very simple models to describe light and gradually increase the complexity but also the power of the model. We start with ray optics and include geometrical optics and lens aberrations. We then move on to scalar waves, introducing Gaussian beams and Fourier optics. The next step is vectorial electromagnetic waves, which allow us to take into account material properties and birefringence. Finally, we come to quantum optics and describe light as a stream of photons.



These notes are 'work in progress', and probably never really finished. If you find mistakes, please tell me. I am also always interested in other sources covering these topics.
The most current version of the lecture notes can be found at github\sidenote{\url{https://github.com/MarkusLippitz}}. There you also find the material for the tasks. I have put everything under a CC-BY-SA license (see footer). In my words: feel free to do with it whatever you like. If you make your work available to the public, mention me and use a similar license. 


The lecture notes are typeset using the LaTeX class 'tufte-book' by Bil Kleb, Bill Wood, and Kevin Godby\sidenote{\href{https://tufte-latex.github.io/tufte-latex/}{tufte-latex}}, which  approximates the work of Edward Tufte\sidenote{\href{https://www.edwardtufte.com/}{edwardtufte.com}}. I applied many of the modifications introduced by Dirk Eddelbuettel in the 'tint' R package\sidenote{\href{https://dirk.eddelbuettel.com/code/tint.html}{tint: Tint is not Tufte}}. For the time being, the source is LaTeX, not markdown.

\vspace{2\baselineskip}

Markus Lippitz \\ Bayreuth, September 18, 2023

