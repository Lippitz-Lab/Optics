\renewcommand{\lastmod}{September 18, 2023}
\renewcommand{\chapterauthors}{Markus Lippitz}

\chapter{Gaussian Beams }

\section{Overview}

%-----------------------------------------------------------------------------
\section{Gauss-Strahlen}

\ziel{Ich kann den Verlauf des elektrischen Feldes in einem Gauß-Fokus \wort{erklären}.}

\begin{itemize}

\item Die Wellengleichung und ihre Lösungen;   ebene, sphärische und paraboloidale Wellen.
\item {Umwandlung einer ebenen in eine sphärische Welle durch eine Linse}
\item {Gauß-Strahlen als paraxiale Lösung der Wellengleichung}
\item {Gauß-Strahlen als Eigenmoden eines Resonators}
\item {Parameter und Eigenschaften von Gauß-Strahlen}
\item Technik: knife edge test
\item {Bonus: Hermit-Gauß’sche Strahlen und Laguerre-Gauß’sche Strahlen}
\item {Bonus: Beziehung zwischen Wellenoptik und Strahlenoptik}

\end{itemize}


\lit{Hering/Martin Kap. 4.6, Saleh/Teich Kap. 2,  3 und 10, Hecht Kap. 13}


%-----------------------------------------------------------------------------
\section{Abbildung von Gauss-Strahlen}

\ziel{Ich kann für typische Linsensysteme den Verlauf eines Gauß-Strahls ‚von Hand‘ \wort{konstruieren} und mittels des ABCD Gesetzes \wort{berechnen}.}

\begin{itemize}
\item {Freiheitsgrade eines Gauss-Strahls}

\item {ABCD Gesetz}
\end{itemize}

\lit{Hering/Martin Kap. 4.6, Saleh/Teich Kap. 3}




%--------------------
\printbibliography[segment=\therefsegment,heading=subbibliography]
